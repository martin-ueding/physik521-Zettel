\input{../header.tex}

%\subject{}
\title{physik521 – Übung 13}
%\subtitle{}
\author{
	Martin Ueding \\ \small{\href{mailto:mu@martin-ueding.de}{mu@martin-ueding.de}}
        \and Paul Manz
        \and Lino Lemmer \\ \small{\href{mailto:l2@uni-bonn.de}{l2@uni-bonn.de}}
}

\pagestyle{plain}

\newcommand\kB{k_\text B}
\newcommand\muB{\mu_\text B}
\newcommand\ZG{Z_\text G}
\DeclareMathOperator{\Tr}{Tr}

\begin{document}

\maketitle

\section{Ising Modell}

\subsection{Molekularfeld-Näherung}

\paragraph{Kanonische Zustandssumme}

Wir berechnen die kanonische Zustandssumme. Wir summieren dazu über alle
Zustände $\ket n$ in der Spur. Diese Zustände bestehen aus den verschiedenen
Spinkonfigurationen der einzelnen Teilchen. Somit haben wir:
\begin{align*}
    Z_\text C
    &= \Tr\sbr{\exp[-\beta \hat H]} \\
    \intertext{%
        Wir setzen den Hamiltonoperator explizit ein:
    }
    &= \Tr\sbr{\exp\sbr{- \beta(J q \bracket\sigma + \mu B) \sum_{i=1}^N
    \sigma_i}}. \\
    \intertext{%
        Wir benutzen als Basis die Spinzustände und schreiben die Spur somit
        als $N$ Summen:
    }
    &= \sum_{\sigma_1 = \pm \frac 12} \ldots \sum_{\sigma_N = \pm \frac 12}
    \exp\sbr{- \beta(J q \bracket\sigma + \mu B) \sum_{i=1}^N \sigma_i}. \\
    \intertext{%
        Die Summe in der Exponentialfunktion ziehen wir raus in ein Produkt. Da
        wir über $N$ Einträge innerhalb der Exponentialfunktion summieren und
        außerhalb ebenfalls über $N$ Einträge summieren, können wir dies in ein
        Produkt zusammenfassen:
    }
    &= \prod_{i=1}^N \sum_{\sigma_i = \pm \frac 12} \exp\sbr{- \beta(J q
    \bracket\sigma + \mu B) \sigma_i}. \\
    \intertext{%
        Das Produkt können wir als einfache Potenz schreiben, da der Faktor
        invariant gegenüber $i$ ist. Diese Summe können wir dann auch noch
        ausführen.
    }
    &= \del{\exp\sbr{\frac12 \beta(J q \bracket\sigma + \mu B)} + \exp\sbr{-
\frac12 \beta(J q \bracket\sigma + \mu B)}}^N \\
    &= 2^N \cosh^N\sbr{\frac12 \beta(J q \bracket\sigma + \mu B)}
\end{align*}

\paragraph{Erwartungswert des Spins}

\begin{align*}
    \bracket{\sigma_j}
    &= \frac{1}{Z_\text C} \Tr[\sigma_j W] \\
    &= \frac{1}{Z_\text C} \prod_{i=1}^N \sum_{\sigma_i = \pm \frac 12}
    \exp\sbr{- \beta(J q \bracket\sigma + \mu B) \sigma_i} \sigma_j \\
    \intertext{%
        Wir sind nur an dem Erwartungswert von einem Spin interessiert. Daher
        setzen wir, ohne Beschränkung der Allgemeinheit, hier $j = 1$. Aus dem
        Produkt ziehen wir den Anteil für $i = 1$ heraus. Die Zustandssumme war
        bereits faktorisiert, so dass wir diese ebenfalls aufteilen können:
    }
    &= \frac{1}{Z_{\text C,1}}
    \sum_{\sigma_1 = \pm \frac 12} \sigma_1
    \exp\sbr{- \beta(J q \bracket\sigma + \mu B) \sigma_1}
    \frac{1}{Z_{\text C,1}^{N-1}}
    \prod_{i=2}^N \sum_{\sigma_i = \pm \frac 12}
    \exp\sbr{- \beta(J q \bracket\sigma + \mu B) \sigma_i}. \\
    \intertext{%
        Vergleicht man den letzten Teil mit der Berechnung der Zustandssumme,
        sieht man, dass dort $Z_{\text C,1}^{N-1}$ herauskommt, sich mit dem
        gleichen Faktor im Nenner kürzt. Es bleibt also:
    }
    &= \frac{1}{Z_{\text C,1}}
    \sum_{\sigma_1 = \pm \frac 12} \sigma_1
    \exp\sbr{- \beta(J q \bracket\sigma + \mu B) \sigma_1}. \\
    \intertext{%
        Ab hier führen wir die Summe aus, setzen $Z_{\text C,1}$ ein und
        vereinfachen mit trigonometrischen Funktionen.
    }
    &= \frac 12\frac{1}{Z_{\text C,1}}
    \del{\exp\sbr{\frac12 \beta(J q \bracket\sigma + \mu B)} - \exp\sbr{-
        \frac12 \beta(J q \bracket\sigma + \mu B)}}  \\
    &= \frac 12 \frac{\exp\sbr{\frac12 \beta(J q \bracket\sigma + \mu B)} -
    \exp\sbr{- \frac12 \beta(J q \bracket\sigma + \mu B)}}{\exp\sbr{\frac12
    \beta(J q \bracket\sigma + \mu B)} + \exp\sbr{- \frac12 \beta(J q
    \bracket\sigma + \mu B)}} \\
    &= \frac 12 \tanh\sbr{\frac12 \beta(J q \bracket\sigma + \mu B)}
\end{align*}

\IfFileExists{\bibliographyfile}{
    \printbibliography
}{}

\end{document}

% vim: spell spelllang=de tw=79
