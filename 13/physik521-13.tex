% Für Seitenformatierung

\documentclass[DIV=15]{scrartcl}

% Zeilenumbrüche

\parindent 0pt
\parskip 6pt

% Für deutsche Buchstaben und Synthax

\usepackage[ngerman]{babel}

% Für Auflistung mit speziellen Aufzählungszeichen

\usepackage{paralist}

% zB für \del, \dif und andere Mathebefehle

\usepackage{amsmath}
\usepackage{commath}
\usepackage{amssymb}

% für nicht kursive griechische Buchstaben

\usepackage{txfonts}

% Für \SIunit[]{} und \num in deutschem Stil

\usepackage[output-decimal-marker={,}]{siunitx}
\usepackage[utf8]{inputenc}

% Für \sfrac{}{}, also inline-frac

\usepackage{xfrac}

% Für Einbinden von pdf-Grafiken

\usepackage{graphicx}

% Umfließen von Bildern

\usepackage{floatflt}

% Für Links nach außen und innerhalb des Dokumentes

\usepackage{hyperref}

% Für weitere Farben

\usepackage{color}

% Für Streichen von z.B. $\rightarrow$

\usepackage{centernot}

% Für Befehl \cancel{}

\usepackage{cancel}

% Für Layout von Links

\hypersetup{
	citecolor=black,
	colorlinks=true,
	linkcolor=black,
	urlcolor=blue,
}

% Verschiedene Mathematik-Hilfen

\newcommand \e[1]{\cdot10^{#1}}
\newcommand\p{\partial}

\newcommand\half{\frac 12}
\newcommand\shalf{\sfrac12}

\newcommand\skp[2]{\left\langle#1,#2\right\rangle}
\newcommand\mw[1]{\left\langle#1\right\rangle}
\renewcommand \exp[1]{\mathrm e^{#1}}

% Nabla und Kombinationen von Nabla

\renewcommand\div[1]{\skp{\nabla}{#1}}
\newcommand\rot{\nabla\times}
\newcommand\grad[1]{\nabla#1}
\newcommand\laplace{\triangle}
\newcommand\dalambert{\mathop{{}\Box}\nolimits}

%Für komplexe Zahlen

\renewcommand \i{\mathrm i}
\renewcommand{\Im}{\mathop{{}\mathrm{Im}}\nolimits}
\renewcommand{\Re}{\mathop{{}\mathrm{Re}}\nolimits}

%Für Bra-Ket-Notation

\newcommand\bra[1]{\left\langle#1\right|}
\newcommand\ket[1]{\left|#1\right\rangle}
\newcommand\braket[2]{\left\langle#1\left.\vphantom{#1 #2}\right|#2\right\rangle}
\newcommand\braopket[3]{\left\langle#1\left.\vphantom{#1 #2 #3}\right|#2\left.\vphantom{#1 #2 #3}\right|#3\right\rangle}

\newcommand{\eqnsep}{,\quad}

\usepackage{tikz}

%\subject{}
\title{physik521 – Übung 13}
%\subtitle{}
\author{
	Martin Ueding \\ \small{\href{mailto:mu@martin-ueding.de}{mu@martin-ueding.de}}
        \and Paul Manz
        \and Lino Lemmer \\ \small{\href{mailto:l2@uni-bonn.de}{l2@uni-bonn.de}}
}

\pagestyle{plain}

\newcommand\kB{k_\text B}
\newcommand\muB{\mu_\text B}
\newcommand\ZG{Z_\text G}
\DeclareMathOperator{\Tr}{Tr}

\begin{document}

\maketitle

\section{Ising Modell}

\subsection{Molekularfeld-Näherung}

\paragraph{Kanonische Zustandssumme}

Wir berechnen die kanonische Zustandssumme. Wir summieren dazu über alle
Zustände $\ket n$ in der Spur. Diese Zustände bestehen aus den verschiedenen
Spinkonfigurationen der einzelnen Teilchen. Somit haben wir:
\begin{align*}
    Z_\text C
    &= \Tr\sbr{\exp[-\beta \hat H]} \\
    \intertext{%
        Wir setzen den Hamiltonoperator explizit ein:
    }
    &= \Tr\sbr{\exp\sbr{- \beta(J q \bracket\sigma + \mu B) \sum_{i=1}^N
    \sigma_i}}. \\
    \intertext{%
        Wir benutzen als Basis die Spinzustände und schreiben die Spur somit
        als $N$ Summen:
    }
    &= \sum_{\sigma_1 = \pm \frac 12} \ldots \sum_{\sigma_N = \pm \frac 12}
    \exp\sbr{- \beta(J q \bracket\sigma + \mu B) \sum_{i=1}^N \sigma_i}. \\
    \intertext{%
        Die Summe in der Exponentialfunktion ziehen wir raus in ein Produkt. Da
        wir über $N$ Einträge innerhalb der Exponentialfunktion summieren und
        außerhalb ebenfalls über $N$ Einträge summieren, können wir dies in ein
        Produkt zusammenfassen:
    }
    &= \prod_{i=1}^N \sum_{\sigma_i = \pm \frac 12} \exp\sbr{- \beta(J q
    \bracket\sigma + \mu B) \sigma_i}. \\
    \intertext{%
        Das Produkt können wir als einfache Potenz schreiben, da der Faktor
        invariant gegenüber $i$ ist. Diese Summe können wir dann auch noch
        ausführen.
    }
    &= \del{\exp\sbr{\frac12 \beta(J q \bracket\sigma + \mu B)} + \exp\sbr{-
\frac12 \beta(J q \bracket\sigma + \mu B)}}^N \\
    &= 2^N \cosh^N\sbr{\frac12 \beta(J q \bracket\sigma + \mu B)}
\end{align*}

\paragraph{Erwartungswert des Spins}

\begin{align*}
    \bracket{\sigma_j}
    &= \frac{1}{Z_\text C} \Tr[\sigma_j W] \\
    &= \frac{1}{Z_\text C} \prod_{i=1}^N \sum_{\sigma_i = \pm \frac 12}
    \exp\sbr{- \beta(J q \bracket\sigma + \mu B) \sigma_i} \sigma_j \\
    \intertext{%
        Wir sind nur an dem Erwartungswert von einem Spin interessiert. Daher
        setzen wir, ohne Beschränkung der Allgemeinheit, hier $j = 1$. Aus dem
        Produkt ziehen wir den Anteil für $i = 1$ heraus. Die Zustandssumme war
        bereits faktorisiert, so dass wir diese ebenfalls aufteilen können:
    }
    &= \frac{1}{Z_{\text C,1}}
    \sum_{\sigma_1 = \pm \frac 12} \sigma_1
    \exp\sbr{- \beta(J q \bracket\sigma + \mu B) \sigma_1}
    \frac{1}{Z_{\text C,1}^{N-1}}
    \prod_{i=2}^N \sum_{\sigma_i = \pm \frac 12}
    \exp\sbr{- \beta(J q \bracket\sigma + \mu B) \sigma_i}. \\
    \intertext{%
        Vergleicht man den letzten Teil mit der Berechnung der Zustandssumme,
        sieht man, dass dort $Z_{\text C,1}^{N-1}$ herauskommt, sich mit dem
        gleichen Faktor im Nenner kürzt. Es bleibt also:
    }
    &= \frac{1}{Z_{\text C,1}}
    \sum_{\sigma_1 = \pm \frac 12} \sigma_1
    \exp\sbr{- \beta(J q \bracket\sigma + \mu B) \sigma_1}. \\
    \intertext{%
        Ab hier führen wir die Summe aus, setzen $Z_{\text C,1}$ ein und
        vereinfachen mit trigonometrischen Funktionen.
    }
    &= \frac 12\frac{1}{Z_{\text C,1}}
    \del{\exp\sbr{\frac12 \beta(J q \bracket\sigma + \mu B)} - \exp\sbr{-
        \frac12 \beta(J q \bracket\sigma + \mu B)}}  \\
    &= \frac 12 \frac{\exp\sbr{\frac12 \beta(J q \bracket\sigma + \mu B)} -
    \exp\sbr{- \frac12 \beta(J q \bracket\sigma + \mu B)}}{\exp\sbr{\frac12
    \beta(J q \bracket\sigma + \mu B)} + \exp\sbr{- \frac12 \beta(J q
    \bracket\sigma + \mu B)}} \\
    &= \frac 12 \tanh\sbr{\frac12 \beta(J q \bracket\sigma + \mu B)}
\end{align*}

\paragraph{Grafische Diskussion}

Für $B = 0$ gilt
\[
    \bracket\sigma = \frac 12 \tanh\sbr{\frac12 \beta J q \bracket\sigma}.
\]

Wir definieren $x := \beta J q \bracket\sigma / 2$ und stellen dies
entsprechend nach $\bracket\sigma$ um. Somit erhalten wir als dimensionslose
Funktionsvorschrift:
\[
    \frac{4 \kB T}{J q} x = \tanh[x].
\]

Dies haben wir in Abbildung~\ref{fig:tanh} grafisch dargestellt. Die kritische
Temperatur ist dann erreicht, wenn die Steigung der Geraden eins wird. Daher
können wir die kritische Temperatur angeben als:
\[
    T_\text C = \frac{Jq}{4 \kB}.
\]

\begin{figure}[htbp]
    \centering
    \begin{tikzpicture}[scale=1.7]
        \clip (-5, -1.5) rectangle (5, 1.5);
        \draw[->, thin] (-3, 0) -- (3, 0) node[right] {$x$};
        \draw[->, thin] (0, -1) -- (0, 1);
        \draw[domain=-3:3, thick] plot (\x, {tanh(\x)}) node[right] {$\tanh[x]$};
        \foreach \m in {0.4, 1, 2}
            \draw[domain=-3:3] plot (\x, {\x * \m});
    \end{tikzpicture}
    \caption{%
        Schnittpunkte von $\tanh[x]$ mit $\frac{4 \kB T}{J q} x$, wobei
        $\frac{4 \kB T}{J q}$ die Werte \numlist{0.4;1;2} annimmt.
    }
    \label{fig:tanh}
\end{figure}

\IfFileExists{\bibliographyfile}{
    \printbibliography
}{}

\end{document}

% vim: spell spelllang=de tw=79
