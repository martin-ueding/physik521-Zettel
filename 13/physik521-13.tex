% Für Seitenformatierung

\documentclass[DIV=15]{scrartcl}

% Zeilenumbrüche

\parindent 0pt
\parskip 6pt

% Für deutsche Buchstaben und Synthax

\usepackage[ngerman]{babel}

% Für Auflistung mit speziellen Aufzählungszeichen

\usepackage{paralist}

% zB für \del, \dif und andere Mathebefehle

\usepackage{amsmath}
\usepackage{commath}
\usepackage{amssymb}

% für nicht kursive griechische Buchstaben

\usepackage{txfonts}

% Für \SIunit[]{} und \num in deutschem Stil

\usepackage[output-decimal-marker={,}]{siunitx}
\usepackage[utf8]{inputenc}

% Für \sfrac{}{}, also inline-frac

\usepackage{xfrac}

% Für Einbinden von pdf-Grafiken

\usepackage{graphicx}

% Umfließen von Bildern

\usepackage{floatflt}

% Für Links nach außen und innerhalb des Dokumentes

\usepackage{hyperref}

% Für weitere Farben

\usepackage{color}

% Für Streichen von z.B. $\rightarrow$

\usepackage{centernot}

% Für Befehl \cancel{}

\usepackage{cancel}

% Für Layout von Links

\hypersetup{
	citecolor=black,
	colorlinks=true,
	linkcolor=black,
	urlcolor=blue,
}

% Verschiedene Mathematik-Hilfen

\newcommand \e[1]{\cdot10^{#1}}
\newcommand\p{\partial}

\newcommand\half{\frac 12}
\newcommand\shalf{\sfrac12}

\newcommand\skp[2]{\left\langle#1,#2\right\rangle}
\newcommand\mw[1]{\left\langle#1\right\rangle}
\renewcommand \exp[1]{\mathrm e^{#1}}

% Nabla und Kombinationen von Nabla

\renewcommand\div[1]{\skp{\nabla}{#1}}
\newcommand\rot{\nabla\times}
\newcommand\grad[1]{\nabla#1}
\newcommand\laplace{\triangle}
\newcommand\dalambert{\mathop{{}\Box}\nolimits}

%Für komplexe Zahlen

\renewcommand \i{\mathrm i}
\renewcommand{\Im}{\mathop{{}\mathrm{Im}}\nolimits}
\renewcommand{\Re}{\mathop{{}\mathrm{Re}}\nolimits}

%Für Bra-Ket-Notation

\newcommand\bra[1]{\left\langle#1\right|}
\newcommand\ket[1]{\left|#1\right\rangle}
\newcommand\braket[2]{\left\langle#1\left.\vphantom{#1 #2}\right|#2\right\rangle}
\newcommand\braopket[3]{\left\langle#1\left.\vphantom{#1 #2 #3}\right|#2\left.\vphantom{#1 #2 #3}\right|#3\right\rangle}

\newcommand{\eqnsep}{,\quad}

\usepackage{tikz}
\usepackage{pdflscape}

%\subject{}
\title{physik521 – Übung 13}
%\subtitle{}
\author{
	Martin Ueding \\ \small{\href{mailto:mu@martin-ueding.de}{mu@martin-ueding.de}}
        \and Paul Manz
        \and Lino Lemmer \\ \small{\href{mailto:l2@uni-bonn.de}{l2@uni-bonn.de}}
}

\pagestyle{plain}

\newcommand\kB{k_\text B}
\newcommand\muB{\mu_\text B}
\newcommand\ZG{Z_\text G}
\DeclareMathOperator{\Tr}{Tr}
\DeclareMathOperator{\sgn}{sgn}

\begin{document}

\maketitle

\section{Ising Modell}

\subsection{Molekularfeld-Näherung}

\paragraph{Kanonische Zustandssumme}

Wir berechnen die kanonische Zustandssumme. Wir summieren dazu über alle
Zustände $\ket n$ in der Spur. Diese Zustände bestehen aus den verschiedenen
Spinkonfigurationen der einzelnen Teilchen. Somit haben wir:
\begin{align*}
    Z_\text C
    &= \Tr\sbr{\exp[-\beta \hat H]} \\
    \intertext{%
        Wir setzen den Hamiltonoperator explizit ein:
    }
    &= \Tr\sbr{\exp\sbr{- \beta(J q \bracket\sigma + \mu B) \sum_{i=1}^N
    \sigma_i}}. \\
    \intertext{%
        Wir benutzen als Basis die Spinzustände und schreiben die Spur somit
        als $N$ Summen:
    }
    &= \sum_{\sigma_1 = \pm \frac 12} \ldots \sum_{\sigma_N = \pm \frac 12}
    \exp\sbr{- \beta(J q \bracket\sigma + \mu B) \sum_{i=1}^N \sigma_i}. \\
    \intertext{%
        Die Summe in der Exponentialfunktion ziehen wir raus in ein Produkt. Da
        wir über $N$ Einträge innerhalb der Exponentialfunktion summieren und
        außerhalb ebenfalls über $N$ Einträge summieren, können wir dies in ein
        Produkt zusammenfassen:
    }
    &= \prod_{i=1}^N \sum_{\sigma_i = \pm \frac 12} \exp\sbr{- \beta(J q
    \bracket\sigma + \mu B) \sigma_i}. \\
    \intertext{%
        Das Produkt können wir als einfache Potenz schreiben, da der Faktor
        invariant gegenüber $i$ ist. Diese Summe können wir dann auch noch
        ausführen.
    }
    &= \del{\exp\sbr{\frac12 \beta(J q \bracket\sigma + \mu B)} + \exp\sbr{-
\frac12 \beta(J q \bracket\sigma + \mu B)}}^N \\
    &= 2^N \cosh^N\sbr{\frac12 \beta(J q \bracket\sigma + \mu B)}
\end{align*}

\paragraph{Erwartungswert des Spins}

\begin{align*}
    \bracket{\sigma_j}
    &= \frac{1}{Z_\text C} \Tr[\sigma_j W] \\
    &= \frac{1}{Z_\text C} \prod_{i=1}^N \sum_{\sigma_i = \pm \frac 12}
    \exp\sbr{- \beta(J q \bracket\sigma + \mu B) \sigma_i} \sigma_j \\
    \intertext{%
        Wir sind nur an dem Erwartungswert von einem Spin interessiert. Daher
        setzen wir, ohne Beschränkung der Allgemeinheit, hier $j = 1$. Aus dem
        Produkt ziehen wir den Anteil für $i = 1$ heraus. Die Zustandssumme war
        bereits faktorisiert, so dass wir diese ebenfalls aufteilen können:
    }
    &= \frac{1}{Z_{\text C,1}}
    \sum_{\sigma_1 = \pm \frac 12} \sigma_1
    \exp\sbr{- \beta(J q \bracket\sigma + \mu B) \sigma_1}
    \frac{1}{Z_{\text C,1}^{N-1}}
    \prod_{i=2}^N \sum_{\sigma_i = \pm \frac 12}
    \exp\sbr{- \beta(J q \bracket\sigma + \mu B) \sigma_i}. \\
    \intertext{%
        Vergleicht man den letzten Teil mit der Berechnung der Zustandssumme,
        sieht man, dass dort $Z_{\text C,1}^{N-1}$ herauskommt, sich mit dem
        gleichen Faktor im Nenner kürzt. Es bleibt also:
    }
    &= \frac{1}{Z_{\text C,1}}
    \sum_{\sigma_1 = \pm \frac 12} \sigma_1
    \exp\sbr{- \beta(J q \bracket\sigma + \mu B) \sigma_1}. \\
    \intertext{%
        Ab hier führen wir die Summe aus, setzen $Z_{\text C,1}$ ein und
        vereinfachen mit trigonometrischen Funktionen.
    }
    &= \frac 12\frac{1}{Z_{\text C,1}}
    \del{\exp\sbr{\frac12 \beta(J q \bracket\sigma + \mu B)} - \exp\sbr{-
        \frac12 \beta(J q \bracket\sigma + \mu B)}}  \\
    &= \frac 12 \frac{\exp\sbr{\frac12 \beta(J q \bracket\sigma + \mu B)} -
    \exp\sbr{- \frac12 \beta(J q \bracket\sigma + \mu B)}}{\exp\sbr{\frac12
    \beta(J q \bracket\sigma + \mu B)} + \exp\sbr{- \frac12 \beta(J q
    \bracket\sigma + \mu B)}} \\
    &= \frac 12 \tanh\sbr{\frac12 \beta(J q \bracket\sigma + \mu B)}
\end{align*}

\paragraph{Grafische Diskussion}

Für $B = 0$ gilt
\[
    \bracket\sigma = \frac 12 \tanh\sbr{\frac12 \beta J q \bracket\sigma}.
\]

Wir definieren $x := \beta J q \bracket\sigma / 2$ und stellen dies
entsprechend nach $\bracket\sigma$ um. Somit erhalten wir als dimensionslose
Funktionsvorschrift:
\[
    \frac{4 \kB T}{J q} x = \tanh[x].
\]

Dies haben wir in Abbildung~\ref{fig:tanh} grafisch dargestellt. Die kritische
Temperatur ist dann erreicht, wenn die Steigung der Geraden eins wird. Daher
können wir die kritische Temperatur angeben als:
\[
    T_\text C = \frac{Jq}{4 \kB}.
\]

\begin{figure}[htbp]
    \centering
    \begin{tikzpicture}[scale=1.7]
        \clip (-5, -1.5) rectangle (5, 1.5);
        \draw[->, thin] (-3, 0) -- (3, 0) node[right] {$x$};
        \draw[->, thin] (0, -1) -- (0, 1);
        \draw[domain=-3:3, thick] plot (\x, {tanh(\x)}) node[right] {$\tanh[x]$};
        \foreach \m in {0.4, 1, 2}
            \draw[domain=-3:3] plot (\x, {\x * \m});
    \end{tikzpicture}
    \caption{%
        Schnittpunkte von $\tanh[x]$ mit $\frac{4 \kB T}{J q} x$, wobei
        $\frac{4 \kB T}{J q}$ die Werte \numlist{0.4;1;2} annimmt.
    }
    \label{fig:tanh}
\end{figure}

\subsection{Bethe-Näherung}

\paragraph{Zustandssumme}

Es gibt einen neuen Hamiltonoperator, als muss die Zustandssumme neu berechnet
werden.
\begin{align*}
    Z_\text C
    &= \sum_{\sigma_0 = \pm \frac 12} \ldots \sum_{\sigma_N = \pm \frac 12}
    \exp\sbr{-\beta \hat H_\text{Bethe} \sbr{\{ \sigma_i \colon i \in \N_0 \}}}
    \\
    &= \sum_{\sigma_0 = \pm \frac 12} \ldots \sum_{\sigma_N = \pm \frac 12}
    \exp\sbr{\beta \del{\mu B \sigma_0 + \mu(B + B') \sum_{j=1}^q \sigma_j + J
    \sum_{j=1}^q \sigma_0 \sigma_j}} \\
    \intertext{%
        Ab hier benutzen wir die Definition von $\alpha_\pm$, die auf dem
        Aufgabenzettel auch benutzt wird:
        \[
            \alpha_\pm = \beta\del{\frac{\mu(B+B')}2 \pm \frac J4}.
        \]
        Damit erhalten wir dann mit dem Signum (Vorzeichenfunktion) $\sgn$:
    }
    &= \sum_{\sigma_0 = \pm \frac 12} \ldots \sum_{\sigma_N = \pm \frac 12}
    \exp\sbr{\beta \mu B \sigma_0 + 2 \alpha_{\sgn[\sigma_0]} \sum_{j=1}^q \sigma_j} \\
    \intertext{%
        Wir schreiben die Summe über $\sigma_0$ abgetrennt. Außerdem trennen
        wir die Summe in der Exponentialfunktion in zwei Faktoren auf.
    }
    &= \sum_{\sigma_0 = \pm \frac 12} \exp[\beta \mu B \sigma_0]
    \sum_{\sigma_1 = \pm \frac 12} \ldots \sum_{\sigma_N = \pm \frac 12}
    \exp\sbr{2 \alpha_{\sgn[\sigma_0]} \sum_{j=1}^q \sigma_j} \\
    &= \sum_{\sigma_0 = \pm \frac 12} \exp[\beta \mu B \sigma_0]
    \sum_{\sigma_1 = \pm \frac 12} \ldots \sum_{\sigma_N = \pm \frac 12}
    \prod_{j=1}^q \exp\sbr{2 \alpha_{\sgn[\sigma_0]} \sigma_j} \\
    \intertext{%
        Nun lassen wir die Summe über $\sigma_i$ auf den Faktor mit $j = 1$
        wirken. Alle anderen Faktoren können wir jeweils ausklammern. Wir
        erhalten damit dann $2 \cosh$ für jedes Summe-Faktor-Paar. Also
        erhalten wir zusammen:
    }
    &= 2^q \sum_{\sigma_0 = \pm \frac 12} \exp[\beta \mu B \sigma_0]
    \cosh^q\sbr{\alpha_{\sgn[\sigma_0]}} \\
    \intertext{%
        Zuletzt führen wir die verbleibende Summe aus.
    }
    &= 2^q \del{ \exp\sbr{\frac 12\beta \mu B} \cosh^q\sbr{\alpha_+}
+ \exp\sbr{- \frac 12\beta \mu B} \cosh^q\sbr{\alpha_-}}
\end{align*}

\paragraph{Erwartungswert von $\sigma_j$}

Warum steht auf dem Aufgabenzettel folgendes:
\[
    \bracket{\sigma_j} = \Bracket{\frac 1q \sum_{j=1}^q \sigma_j} ?
\]
Auf der linken Seite ist der Index $j$ frei, auf der rechten Seite wird über
ihn summiert, daher ist er nicht mehr frei und nach außen sichtbar. Links steht
ein Vektor, rechts ein Skalar. Sollen wir links den Index weglassen? Oder soll
man die linke Seite ignorieren und den Erwartungswert für $\sigma_j$
ausrechnen, so wie wir unten den Erwartungswert für $\sigma_0$ ausrechnen?

\fehlt

\paragraph{Erwartungswert von $\sigma_0$}

\begin{align*}
    \bracket{\sigma_0}
    &= \frac{1}{Z_\text C} \sum_{\sigma_0 = \pm\frac12} \ldots \sum_{\sigma_q = \pm\frac12}
    \exp\sbr{\beta \mu B \sigma_0 + 2 \alpha_{\sgn[\sigma_0]} \sum_{j=1}^q
    \sigma_j} \sigma_0 \\
    &= \frac{1}{Z_\text C} \sum_{\sigma_0 = \pm\frac12} \ldots \sum_{\sigma_q = \pm\frac12}
    \exp[\beta \mu B \sigma_0] 
    \exp\sbr{2 \alpha_{\sgn[\sigma_0]} \sum_{j=1}^q
    \sigma_j} \sigma_0 \\
    \intertext{%
        Jetzt können wir die ganzen Summanden $\sigma_j$ wieder als Produkte
        aus der Exponentialfunktion ziehen. Somit erhalten wir jeweils $2
        \cosh$.
    }
    &= \frac{1}{Z_\text C} \sum_{\sigma_0 = \pm\frac12}
    \exp[\beta \mu B \sigma_0] 
    2^q \cosh^q\sbr{\alpha_{\sgn[\sigma_0]}} \sigma_0 \\
    \intertext{%
        Wir führen die letzte Summe aus.
    }
    &= \frac{2^q}{2 Z_\text C} \del{
        \exp\sbr{\frac 12 \beta \mu B \sigma_0} \cosh^q[\alpha_+]
        - \exp\sbr{-\frac 12 \beta \mu B \sigma_0} \cosh^q[\alpha_-]
    } \\
    \intertext{%
        Mit der Definition von $\tilde Z_\text C$ erhalten wir genau den
        Ausdruck auf dem Aufgabenblatt:
    }
    &= \frac{1}{2 \tilde Z_\text C} \del{
        \exp\sbr{\frac 12 \beta \mu B \sigma_0} \cosh^q[\alpha_+]
        - \exp\sbr{-\frac 12 \beta \mu B \sigma_0} \cosh^q[\alpha_-]
    }.
\end{align*}


\paragraph{Gleichheit der Gitterplätze}

Es soll $\bracket{\sigma_0} = \bracket{\sigma_j}$ gelten. Also setzen wir das
gleich. Dabei gehen wir von den Kontrollergebnissen aus. Hier kommt dann der
Punkt, an dem man das Papier quer nehmen muss …
\begin{landscape}
\begin{align*}
    \exp\sbr{\frac{\beta \mu B}2} \cosh^{q-1}[\alpha_+] \sinh[\alpha_+]
    + \exp\sbr{- \frac{\beta \mu B}2} \cosh^{q-1}[\alpha_-] \sinh[\alpha_-]
    &=
    \exp\sbr{\frac{\beta \mu B}2} \cosh^q[\alpha_+]
    - \exp\sbr{- \frac{\beta \mu B}2} \cosh^q[\alpha_-] \\
    \intertext{%
        Wir setzen $B = 0$. Dadurch werden alle Exponentialfunktionen 1 und
        fallen heraus.
    }
    \cosh^{q-1}[\alpha_+] \sinh[\alpha_+] + \cosh^{q-1}[\alpha_-] \sinh[\alpha_-]
    &= \cosh^q[\alpha_+] - \cosh^q[\alpha_-] \\
    \cosh^{q-1}[\alpha_+] \del{\sinh[\alpha_+] - \cosh[\alpha_+]} + \cosh^{q-1}[\alpha_-] \del{
    \sinh[\alpha_-] + \cosh[\alpha_-]} &= 0 \\
    \intertext{%
        Die Klammern können wir mit
        \[
            \sinh[x] - \cosh[x] = \frac 12 \del{ (\eup^x - \eup^{-x}) - (\eup^x
            + \eup^{-x})} = - \eup^{-x}
            \qquad \text{und} \qquad
            \sinh[x] + \cosh[x] = \frac 12 \del{ (\eup^x - \eup^{-x}) + (\eup^x
            + \eup^{-x})} = \eup^{x}.
        \]
        vereinfachen.
    }
    - \cosh^{q-1}[\alpha_+] \exp[-\alpha_+] + \cosh^{q-1}[\alpha_-] \exp[\alpha_-] &= 0 \\
    \intertext{%
        Jetzt ist die Gleichung schon wieder so kompakt, dass sie eigentlich
        gar nicht mehr ins Querformat muss. Daher setzen wir jetzt $\alpha_\pm$
        ein. An der Stelle $B = 0$ ist $\alpha_\pm$ gegeben durch $\beta \mu
        B'/2 + \beta J/4$.
    }
    - \cosh^{q-1}\sbr{\frac{\beta \mu B'}4 + \frac{\beta J}4}
    \exp\sbr{- \frac{\beta \mu B'}4 - \frac{\beta J}4}
    + \cosh^{q-1}\sbr{\frac{\beta \mu B'}4 - \frac{\beta J}4}
    \exp\sbr{\frac{\beta \mu B'}4 - \frac{\beta J}4} &= 0 \\
    %
    \exp\sbr{- \frac{\beta J}4} \del{
        - \cosh^{q-1}\sbr{\frac{\beta \mu B'}4 + \frac{\beta J}4}
        \exp\sbr{- \frac{\beta \mu B'}4}
        + \cosh^{q-1}\sbr{\frac{\beta \mu B'}4 - \frac{\beta J}4}
        \exp\sbr{\frac{\beta \mu B'}4}
    } &= 0 \\
    \intertext{%
        Damit die Exponentialfunktion null werden kann, muss $\beta \to \infty$
        gehen. Dies ist allerdings keine endliche, kritische Temperatur und
        wird daher nicht weiter beachtet.
    }
    - \cosh^{q-1}\sbr{\frac{\beta \mu B'}4 + \frac{\beta J}4}
    \exp\sbr{- \frac{\beta \mu B'}4}
    + \cosh^{q-1}\sbr{\frac{\beta \mu B'}4 - \frac{\beta J}4}
    \exp\sbr{\frac{\beta \mu B'}4} &= 0 \\
\end{align*}
\end{landscape}

Ab hier kommen wir ohne Näherungen wahrscheinlich nicht mehr weiter. Daher
entwickeln wir ab hier. In der Aufgabenstellung ist vorgegeben, dass man um $B'
= 0$ entwickeln soll. Wir gehen davon aus, dass $B' = 0$ ist, weil wir gerade
am Phasenübergang interessiert sind. An diesem tritt die spontane
Symmetriebrechung aus und wir erhalten eine makroskopische Magnetisierung durch
die Spins. $B'$ kann hier wahrscheinlich als Ordnungsparameter benutzt werden.
Am Phasenübergang ist diese Ordnung jedoch noch nicht sonderlich ausgeprägt,
sondern erscheint oder verschwindet gerade.

Wir setzen in die letzt Zeile der vorherigen Seite $B' = 0$ ein und erhalten
\[
    - \cosh\sbr{\frac{\beta J}4}^{q-1} + \cosh\sbr{-\frac{\beta J}4}^{q-1} = 0,
\]
was immer erfüllt ist, da $\cosh$ eine gerade Funktion ist.

Als nächstes bildet wir die erste Ableitung der Gleichung. Das folgende ist
eine Gleichung, die wir einfach in vier Zeilen zerlegen mussten. Dies ändert
aber nichts an der Bedeutung.
\begin{multline*}
    \del{
        - (q-1) \cosh\sbr{\frac{\beta \mu B'}2 + \frac{\beta J}4}^{q-2}
        \sinh\sbr{\frac{\beta \mu B'}2 + \frac{\beta J}4}
        \frac{\beta \mu}2
        - \cosh\sbr{\frac{\beta \mu B'}2 + \frac{\beta J}4}^{q-1}
        \del{- \frac{\beta \mu}2}
    } \\
    \exp\sbr{- \frac{\beta \mu B'}2} \\
    +
    \del{
        (q-1) \cosh\sbr{\frac{\beta \mu B'}2 - \frac{\beta J}4}^{q-2}
        \sinh\sbr{\frac{\beta \mu B'}2 - \frac{\beta J}4}
        \frac{\beta \mu}2
        + \cosh\sbr{\frac{\beta \mu B'}2 - \frac{\beta J}4}^{q-1}
        \frac{\beta \mu}2
    }
    \\
    \exp\sbr{\frac{\beta \mu B'}2}
    = 0
\end{multline*}

Dort setzen wir auch $B'=0$ ein und erhalten:
\begin{multline*}
    - (q-1) \cosh\sbr{\frac{\beta J}4}^{q-2}
    \sinh\sbr{\frac{\beta J}4}
    \frac{\beta \mu}2
    - \cosh\sbr{\frac{\beta J}4}^{q-1}
    \del{- \frac{\beta \mu}2}
    \\
    +
    (q-1) \cosh\sbr{- \frac{\beta J}4}^{q-2}
    \sinh\sbr{- \frac{\beta J}4}
    \frac{\beta \mu}2
    + \cosh\sbr{- \frac{\beta J}4}^{q-1}
    \frac{\beta \mu}2
    = 0
\end{multline*}

Nun können wir dies ein wenig vereinfachen, indem wir wieder ausnutzen, dass
$\cosh$ gerade und $\sinh$ ungerade ist. Den Faktor 2 teilen wir heraus:
\[
    \del{
        - (q-1) \cosh\sbr{\frac{\beta J}4}^{q-2}
        \sinh\sbr{\frac{\beta J}4}
        + \cosh\sbr{\frac{\beta J}4}^{q-1}
    }
    \frac{\beta \mu}2
    = 0
\]

\paragraph{Kritische Temperatur}

Nun ist eine Lösung dieser Gleichung, dass $\beta = 0$ ist. Dies ist jedoch
unrealistisch. Daher muss die Klammer null sein.
\begin{align*}
    (q-1) \cosh\sbr{\frac{\beta J}4}^{q-2}
    \sinh\sbr{\frac{\beta J}4}
    - \cosh\sbr{\frac{\beta J}4}^{q-1} &= 0 \\
    %
    \cosh\sbr{\frac{\beta J}4}^{q-2}
    \del{
        (q-1) \sinh\sbr{\frac{\beta J}4}
        - \cosh\sbr{\frac{\beta J}4}
    } &= 0 \\
    \intertext{%
        Da $\cosh$ positiv definit ist, kann dieser Faktor nicht null sein.
        Daher muss es wieder die Klammer sein.
    }
    (q-1) \sinh\sbr{\frac{\beta J}4}
    - \cosh\sbr{\frac{\beta J}4} &= 0 \\
    \frac 4J \arcoth[q-1] &= \beta \\
    \frac{J}{4 \kB} \arcoth[q-1]^{-1} &= T_\text C
\end{align*}

Wie in Abbildung~\ref{fig:arcoth} zu sehen, verhält sich für $q\to\infty$
dieser Term genauso, wie die Skalierung mit $\propto q$ in der vorherigen
Teilaufgabe. Damit erhalten wir das gleiche Ergebnis.

\begin{figure}[htbp]
    \centering
    \begin{tikzpicture}
        \draw[->, thin] (0, 0) -- (4, 0) node[right] {$x$};
        \draw[->, thin] (0, 0) -- (0, 4);
        \draw[domain=1.1:4, thick] plot (\x, {1/(ln((\x+1)/(\x-1))/2)})
        node[right] {$\arcoth[x]^{-1}$};
        \draw[domain=1.1:4] plot (\x, {\x}) node[left] {$x$};
    \end{tikzpicture}
    \caption{%
        Asymptotisches Verhalten der Funktion $\arcoth[x]^{-1}$.
    }
    \label{fig:arcoth}
\end{figure}


\section{Ginzburg-Landau Theorie des Heisenberg-Modells}

\fehlt

\IfFileExists{\bibliographyfile}{
    \printbibliography
}{}

\end{document}


