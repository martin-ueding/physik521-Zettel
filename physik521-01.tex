\input{header.tex}

\setcounter{section}{1}
\renewcommand\thesection{H\,1.\arabic{section}}

\title{physik521: Übungsblatt 01}
\author{Lino Lemmer \and Martin Ueding \\ \small{\texttt{mu@martin-ueding.de}}}

\begin{document}
\maketitle
\section{Maxwell Relationen und Ableitungsregeln}

\begin{enumerate}[(a)]
    \item
    \item
    \item
\end{enumerate}

\section{Wahrscheinlichkeitstheorie}

\begin{enumerate}[(a)]
    \item
        \[
            X = \text{Anzahl der geraden Augenzahlen}
        \]
    \item
        Die Wahrscheinlichkeitsverteilung von $X$ ist
        \begin{align*}
            P_X(0) &= \half\cdot\half = \frac 14\\
            P_X(1) &= \half\cdot\half + \half\cdot\half = \half\\
            P_X(2) &= \half\cdot\half = \frac 14\\
            \sum P_X &= \frac 14 + \half + \frac 14 = 1
        \end{align*}
    \item
        Der Mittelwert zur Zufallsvariable $X$ ist
        \[
            \mw{X} = 0\cdot\frac14 + 1\cdot\half + 2\cdot\frac14 = 1
        \]
    \item
        Das Schwankungsquadrat von $X$ ist
        \begin{align*}
            \Delta X &= \sqrt{\mw{\del{X - 1}^2}}\\
                       &= \sqrt{\frac 14 \cdot 1 + \half \cdot 0 + \frac 14 \cdot 1}\\
                       &= \frac1{\sqrt{2}}
        \end{align*}
    \item
    \item
    \item
\end{enumerate}
\end{document}
