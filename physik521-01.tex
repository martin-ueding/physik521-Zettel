\input{header.tex}

\setcounter{section}{1}
\renewcommand\thesection{H\,1.\arabic{section}}
\renewcommand\thesubsection{\thesection.\alph{subsection}}

\title{physik521: Übungsblatt 01}
\author{Lino Lemmer \and Martin Ueding \\ \small{\texttt{mu@martin-ueding.de}}}

\begin{document}
\maketitle
\section{Maxwell Relationen und Ableitungsregeln}


\section{Wahrscheinlichkeitstheorie}

\subsection{Zufallsvariable}

\[
    X = \text{Anzahl der geraden Augenzahlen}
\]

\subsection{Wahrscheinlichkeitsverteilung}

Die Wahrscheinlichkeitsverteilung von $X$ ist
\begin{align*}
    P_X(0) &= \half\cdot\half = \frac 14\\
    P_X(1) &= \half\cdot\half + \half\cdot\half = \half\\
    P_X(2) &= \half\cdot\half = \frac 14\\
    \sum P_X &= \frac 14 + \half + \frac 14 = 1
\end{align*}

\subsection{Mittelwert}

Der Mittelwert zur Zufallsvariable $X$ ist
\[
    \mw{X} = 0\cdot\frac14 + 1\cdot\half + 2\cdot\frac14 = 1
\]

\subsection{Schwankungsquadrat}

        Das Schwankungsquadrat von $X$ ist
        \begin{align*}
            \Delta X &= \sqrt{\mw{\del{X - 1}^2}}\\
                     &= \sqrt{\frac 14 \cdot 1 + \half \cdot 0 + \frac 14 \cdot 1}\\
                     &= \frac1{\sqrt{2}}
        \end{align*}

\subsection{Korrelationsfunktion}
\subsubsection{Würfel}
Ich berechne zunächst die Mittelwerte für den Würfel
\begin{align*}
    \mw{X_1^1} &= \mw{X_2^1} \\
               &= \frac 16 \del{1+2+3+4+5+6} \\
               &= \num{3.5} \\
    \intertext{Die Korrelationsfunktion ist}
    K_{12} &= \mw{ \del{ X_1^1 - \mw{X_1^1} } \del{ X_2^1 - \mw{X_2^1}}} \\
           &= \mw{ \del{ X_1^1 - \num{3.5}}\del{ X_2^1 - \num{3.5}}}
    \intertext{Da genauso viel negative Abweichungen von den Mittelwerten wie positive Auftauchen, folgt}
    &= 0
\end{align*}
Die Zufallsgrößen sind also unkorreliert.

\subsubsection{Fußball-Physiker}

    Nun für den Fußball-Physiker
    \begin{align*}
    \mw{X_1^2} &= \\
    \mw{X_2^2} &=
    \end{align*}
\end{document}
