% Für Seitenformatierung

\documentclass[DIV=15]{scrartcl}

% Zeilenumbrüche

\parindent 0pt
\parskip 6pt

% Für deutsche Buchstaben und Synthax

\usepackage[ngerman]{babel}

% Für Auflistung mit speziellen Aufzählungszeichen

\usepackage{paralist}

% zB für \del, \dif und andere Mathebefehle

\usepackage{amsmath}
\usepackage{commath}
\usepackage{amssymb}

% für nicht kursive griechische Buchstaben

\usepackage{txfonts}

% Für \SIunit[]{} und \num in deutschem Stil

\usepackage[output-decimal-marker={,}]{siunitx}
\usepackage[utf8]{inputenc}

% Für \sfrac{}{}, also inline-frac

\usepackage{xfrac}

% Für Einbinden von pdf-Grafiken

\usepackage{graphicx}

% Umfließen von Bildern

\usepackage{floatflt}

% Für Links nach außen und innerhalb des Dokumentes

\usepackage{hyperref}

% Für weitere Farben

\usepackage{color}

% Für Streichen von z.B. $\rightarrow$

\usepackage{centernot}

% Für Befehl \cancel{}

\usepackage{cancel}

% Für Layout von Links

\hypersetup{
	citecolor=black,
	colorlinks=true,
	linkcolor=black,
	urlcolor=blue,
}

% Verschiedene Mathematik-Hilfen

\newcommand \e[1]{\cdot10^{#1}}
\newcommand\p{\partial}

\newcommand\half{\frac 12}
\newcommand\shalf{\sfrac12}

\newcommand\skp[2]{\left\langle#1,#2\right\rangle}
\newcommand\mw[1]{\left\langle#1\right\rangle}
\renewcommand \exp[1]{\mathrm e^{#1}}

% Nabla und Kombinationen von Nabla

\renewcommand\div[1]{\skp{\nabla}{#1}}
\newcommand\rot{\nabla\times}
\newcommand\grad[1]{\nabla#1}
\newcommand\laplace{\triangle}
\newcommand\dalambert{\mathop{{}\Box}\nolimits}

%Für komplexe Zahlen

\renewcommand \i{\mathrm i}
\renewcommand{\Im}{\mathop{{}\mathrm{Im}}\nolimits}
\renewcommand{\Re}{\mathop{{}\mathrm{Re}}\nolimits}

%Für Bra-Ket-Notation

\newcommand\bra[1]{\left\langle#1\right|}
\newcommand\ket[1]{\left|#1\right\rangle}
\newcommand\braket[2]{\left\langle#1\left.\vphantom{#1 #2}\right|#2\right\rangle}
\newcommand\braopket[3]{\left\langle#1\left.\vphantom{#1 #2 #3}\right|#2\left.\vphantom{#1 #2 #3}\right|#3\right\rangle}

\newcommand{\eqnsep}{,\quad}


\hypersetup{
	pdftitle=
}

%\subject{}
\title{physik521 -- Übung 9}
%\subtitle{}
\author{
	Martin Ueding \\ \small{\href{mailto:mu@martin-ueding.de}{mu@martin-ueding.de}}
        \and
        Paul Manz
        \and
        Lino Lemmer
}

\pagestyle{plain}

\begin{document}

\maketitle

\section{Das ideale Fermi-Gas}

\newcommand\kB{k_\text B}
\newcommand\ZG{Z_\text G}
\newcommand\ZGe{Z_{\text G, 1}}
\newcommand\Eai{E_{\alpha_i}}
\newcommand\isum{\sum_{i=1}^\infty }

Wir übernehmen folgende Formeln aus dem Skript:
\begin{gather*}
    \ZG = \prod_{i=1}^\infty \del{1 + \exp\del{- \frac{\Eai - \mu}{\kB T}}} \\
    \Omega = - \kB T \isum \ln \del{\ZGe(\alpha_i, \mu, T)} \\
    \rho(E) = \isum \delta(E - \Eai)
\end{gather*}

Alle diese Aufgaben stehen so im Skript vorgerechnet. Wir haben uns bei
unseren Rechnungen an das Skript gehalten, jedoch wieder versucht jeden
Schritt hier zur erklären.

\subsection{Allgemeine Ausdrücke}

\subsubsection{Mittlere Teilchenzahl}

Die Funktion $f(\Eai)$ gibt die Anzahl der Besetzungen des Zustandes $\alpha_i$ an. Somit wie im Skript:
\[
    \bracket N = \isum f(\Eai).
\]

Wenn man jetzt $\rho$ dazunimmt, geht dies so:
\[
    \bracket N = \int \dif E \, \rho(E) f(E).
\]

Dies funktioniert, da $\rho$ im Integral über $E$ an jedem Energiewert $E$ die
Anzahl der Zustände $\alpha_i$, die diese Energie haben, gibt. Somit wird die
Vielfachheit der Energiezustände von der Summe in das $\rho$ verlagert.

\subsubsection{Entropie}

Mit $\Omega = U - TS - \mu N$ kann man die Entropie $S$ als Ableitung schreiben:
\[
    S = - \tdpd\Omega T{\mu, V}.
\]

Diese rechnen wir jetzt konkret aus:
\begin{align*}
    S &= \kB \isum \ln\del{- \frac{\Eai - \mu}{\kB T}} + \kB T \isum
    \frac{\exp\del{- \frac{\Eai - \mu}{\kB T}}}{1 + \exp\del{- \frac{\Eai -
    \mu}{\kB T}}} \frac{\Eai - \mu}{\kB T^2} \\
      &= - \frac\Omega T + \isum f(\Eai) \frac{\Eai - \mu}T.
\end{align*}

\subsubsection{Innere Energie}

Mit der am Anfang der vorherigen Teilaufgabe genannten Relation können wir die innere Energie als $U = \Omega + TS + \mu N$ schreiben. Wir rechnen weiter:
\begin{align*}
    U &= \Omega + TS + \mu N \\
      &= \Omega - \Omega + \isum f(\Eai) \cdot (\Eai - \mu) + \mu N \\
      &= \isum f(\Eai) \cdot \Eai + \isum f(\Eai) \cdot \Eai + \mu N \\
      &= \bracket E - \bracket N \mu + \mu N. \\
    \intertext{%
        Wenn man das Limit $N \to \infty$ annimmt, kann man $\lim_{N\to\infty}
        \bracket N = N$ annehmen und weiter vereinfachen:
    }
      &= \bracket E.
\end{align*}

\subsubsection{Freie Energie}

Die freie Energie $F$ ist wieder ein transformiertes Potential, wir benutzen
hier $F = \Omega + \mu N$. Dabei sehen wir jedoch nicht, wie sich dies weiter
vereinfacht.

\subsection{Druck des Gases}

Im Skript wird folgendes vorgerechnet. In Formel (5.47) ist das großkanonische Potential gegeben:
\[
    \Omega = - \kB T \sum_\sigma \dif E_\alpha \rho_\sigma(E_\alpha) \ln\del{1 + \exp\del{- \frac{E_\alpha - \mu}{\kB T}}}.
\]

Dabei stehen die $\sigma$ wahrscheinlich für alle Spins, über die Summiert wird. Allerdings ist der Index $\alpha$ frei, sollte aber wegsummiert werden, da er auf der linken Seite nicht steht. Fehlt somit noch eine Summe über $\alpha$, oder bezieht die Summe über die $\sigma$ dies noch ein?

\newcommand\EF{\epsilon_\text F}

In Formel (5.28) ist die Zustandsdichte pro Volumen gegeben:
\[
    \frac{\rho_\sigma(E)} V = \frac 34 n \frac1\EF \sqrt{\frac E\EF}.
\]

Wenn man dies jetzt nach $\rho_\sigma$ umstellt und in die obige Formel für $\Omega$ einsetzt, und noch annimmt, dass die Abhängigkeit vom Volumen jetzt alleine in dem $V$ steckt, das durch die Umformung entstanden ist, führt eine Ableitung nach $V$ nur dazu, dass dieses $V$ wieder verschwindet. Somit erhalten wir das in Skript gegebene Ergebnis von:
\[
    p = - \dpd \omega V = - \frac \Omega V.
\]

Jetzt haben wir dabei wohl angenommen, dass die Teilchendichte $n = N/V$ konstant bleibt, weil $\rho_\sigma/V$ diese enthält und wir nur so diese proportionale Volumenabhängigkeit erhalten. Allerdings wurde das $\mu$ in der Darstellung von $\Omega$ als konstant angenommen, damit die Ableitung derart einfach ist.

Von daher scheint die Herleitung im Skript beide Annahmen gemacht zu haben.

\subsubsection{Bei festem chemischen Potential}

\fehlt

\subsubsection{Bei fester Teilchenzahl}

\fehlt

\subsubsection{Diskussion}

\fehlt

\subsection{Spezifische Wärme}

Die Definition der spezifischen Wärme ist:
\[
    c_V = \tdpd QTV.
\]

Hier ist die Variante mit
\[
    c_V = - T \tdpd ST{V,\mu}
\]
nützlich. Dort setzen wir jetzt die Entropie, die wir vorher berechnet haben, ein:
\begin{align*}
    c_V &= -T \sbr{
    \frac{\Omega}{T^2} - \isum f(\Eai) \frac{\Eai - \mu}{T^2} - \frac 1T \tdpd \Omega T {V,\mu} - \isum \tdpd{f(\Eai)}T{V,\mu} \frac{\Eai - \mu}T
    } \\
    &= - \frac\Omega T + \isum f(\Eai) \frac{\Eai - \mu}T - \tdpd\Omega T{V,\mu} + \isum \tdpd{f(\Eai)}T{V,\mu} (\Eai - \mu). \\
    \intertext{%
        Die ersten beiden Summanden sind gerade die Entropie. Der dritte
        Summand ist $-S$. Somit erhalten wir:
    }
    &= \isum \tdpd{f(\Eai)}T{V,\mu} (\Eai - \mu). \\
    \intertext{%
        Wir ziehen die Ableitung nach vorne, da die restlichen Größen als
        unabhängig von der Temperatur angenommen werden.
    }
    &= \tdpd{}T{V,\mu} \isum f(\Eai) \cdot (\Eai - \mu) \\
    \intertext{%
        Es bleibt der Mittelwert von $\Eai - \mu$ übrig. Somit schreiben wir:
    }
    &= \tdpd{}T{V,\mu} \bracket{\Eai - \mu} \\
    \intertext{%
        Dies ist der Mittelwert der Abweichung der Energiewerte vom chemischen
        Potential. In der zeitlichen Ableitung spielt $\mu$ jedoch keine Rolle
        mehr, da es als konstant angenommen wird. Somit ist es nur der
        Mittelwert der Energie.
    }
    &= \tdpd{\bracket\Eai}T{V,\mu} \\
    \intertext{%
        Dies identifizieren wir mit der inneren Energie:
    }
    &= \tdpd{U}T{V,\mu}.
\end{align*}

\subsection{Tieftemperaturentwicklung}

Für die weiteren Teilaufgaben muss das großkanonische Potential genähert
werden. Dabei wird ausgenutzt, dass bei kleinen Temperaturen die
Fermiverteilung jenseits des chemischen Potentials annähernd konstant ist.

\subsubsection{Mittlere Teilchenzahl}

\begin{align*}
    N &= 2 \int \dif E \, \rho(E) f(E) \\
    \intertext{%
        Nun führen wir eine partielle Integration durch, um die Ableitung der
        Fermiverteilung zu erhalten. Dazu müssen wir dann eine Funktion
        \[
            b(E) := \int_{-\infty}^E \dif E' \, \rho(E')
        \]
        einführen, die die Stammfunktion zu $\rho$ ist.
    }
    &= 2 \eval{b(E) f(E)}_{E=-\infty}^\infty - 2 \int \dif E \, b(E) f'(E) \\
    \intertext{%
        Der erste Summand verschwindet, da $b(-\infty)$ per Definition
        verschwindet. $f(\infty)$ verschwindet auch, da große Energien immer
        weniger dicht besetzt sind.
    }
    &= - 2 \int \dif E \, b(E) f'(E) \\
    \intertext{%
        Da $f'(E)$ an einer recht schmalen Stelle bei $E = \mu$ von null
        verschieden ist, und $b$ innerhalb des relevanten Intervalls nicht
        stark variiert, kann es durch eine Taylorentwicklung angenähert werden.
        Diese ist:
        \[
            b(E) = b(\mu) + b'(\mu) (E-\mu) + \frac 12 b''(\mu) (E-\mu)^2 +
            \mathcal O (E^3).
        \]
        Wir setzen diese Entwicklung für $b$ ein und erhalten:
    }
    &= - 2 b(\mu) \int \dif E \, f'(E) - 2 b'(\mu) \int \dif E \, (E-\mu)
    f'(E) - b''(\mu) \int \dif E \, (E-\mu)^2 f'(E). \\
    \intertext{%
        Im Skript sind die Lösungen zu diesen sogenannten Fermiintegralen
        angeben. Diese setzen wir hier ein.
    }
    &= 2 b(\mu) + p'(\mu) \frac{\pi^2}{3} (\kB T)^2
\end{align*}

$b(\mu)$ ist die Anzahl der Zustände, die bis zum chemischen Potential besetzt
sind.

\subsubsection{Entropie}

Im Skript wird eine analoge Näherung des kanonischen Potentials gemacht. Da im
kanonischen Potential zuerst der Logarithmus der Zustandssumme steht, braucht
man eine weitere partielle Integration, um die Ableitung der Fermiverteilung
zu erhalten. Der Ausdruck ist gegeben als:
\[
    \Omega = -2 b(\mu) - \frac{\pi^2}3 p(\mu) (\kB T)^2 + \mathcal O(T^4).
\]

Die Entropie ist die negative Ableitung nach der Temperatur, also:
\[
    S = - \dpd \Omega T = \frac 23 \pi^2 \rho(\mu) \kB^2 T.
\]

\subsubsection{Innere Energie}

Die innere Energie ist $U = \Omega + TS + \mu N$. Mit den Ergebnissen der
vorherigen Teilaufgaben:
\[
    U = - 2 b(\mu) + \frac{\pi^2}3 \rho(\mu) (\kB T)^2 + \mu N.
\]

Wenn man $b(\mu) = N$ annimmt, das allerdings nicht exakt sein kann, weil wir
in der vorherigen Aufgabe etwas anderes für $\bracket N$ erhalten haben,
könnte man dies vielleicht noch weiter vereinfachen.

\subsubsection{Freie Energie}

Die freie Energie ist, wie zuvor auch, $\Omega + \mu N$ und lässt sich nicht
sonderlich vereinfachen.

\subsubsection{Druck}

Der Druck ist weiterhin
\[
    p = - \dpd \Omega V,
\]
wobei das $\Omega$ jetzt das genäherte ist. Da $\rho(\omega)$ wieder
proportional von $V$ ist und der erste Summand in $\Omega$ gar nicht von $V$
abhängt, ist der Druck gegeben durch:
\[
    p = \frac{\piup^2}6 \frac{\rho(\mu)}V (\kB T)^2.
\]

\section{Modell für weiße Zwerge}

\subsection{Fermiimpuls}
Für ein ideales Fermigas gilt die Gleichung 
\[ \bracket N = \int \dif E \ \rho(E) f_\mu(E) \]
Wir wissen, dass bei $T=0$ alle Zustände bis zur Fermienergie besetzt sind. Um diese und damit den Fermiimpuls zu erhalten brauchen wir die Gleichung lediglich bei $T=0 \implies f_\mu=\Theta(\epsilon_{\text F})$ auszuwerten.
\begin{align*}
N &= \int_0^{\epsilon_{\text F}} \dif E \ \rho(E) \\
&= \int_0^{\epsilon_{\text F}} \dif E \ \frac{VE \sqrt{E^2-(mc^2)^2}}{2\pi^2\hbar^3c^3} \\
&= \frac{V}{2\pi^2\hbar^3c^3} \int_0^{p_{\text F}} \dif p \ \frac{Ep^2c^3}{2E} \\
&= \frac{V}{4\pi^2\hbar^3c} \int_0^{p_{\text F}} \dif p \ p^2 \\
&= \frac{Vp_{\text F}^3}{12\pi^2\hbar^3} \\
\implies p_{\text F} &= \hbar \ \sqrt[3]{12\pi^2 n}
\end{align*}

Damit Elektronen sich an der Fermikante relativistisch bewegen muss $p_{\text F} > mc$ gelten:
\begin{align*}
p_{\text F} &> mc \\
\implies \hbar \ \sqrt[3]{12\pi^2 n} &> mc \\
\implies n &> \frac{(mc)^3}{12\pi^2\hbar^3} \\
n &> \SI{1.49e29}{\centi \meter}^{-3}
\end{align*}

\subsection{Temperatur}
Berechne die Fermienergie mithilfe der relativistischen Energie-Impuls-Beziehung.
\begin{align*}
\epsilon_{\text F} &= c \sqrt{m^2c^2+p_{\text F}^2} \\
&= c \sqrt{m^2c^2+\hbar^2 (12\pi^2 n)^{2/3}} \\
&= \SI{1.75e-13}{\joule}
\end{align*}
Für $k_{\text B}T$ erhalten wir:
\begin{align*}
k_{\text B}T = \SI{1.38e-16}{\joule} \ll \epsilon_{\text F}
\end{align*}

\subsection{Innere Energie}

\IfFileExists{\bibliographyfile}{
    \printbibliography
}{}

\end{document}

% vim: spell spelllang=de
