\input{header.tex}

%\subject{}
\title{physik521 – Übung 10}
%\subtitle{}
\author{
	Martin Ueding \\ \small{\href{mailto:mu@martin-ueding.de}{mu@martin-ueding.de}}
        \and Paul Manz
        \and Lino Lemmer \\ \small{\href{mailto:l2@uni-bonn.de}{l2@uni-bonn.de}}
}

\pagestyle{plain}

\begin{document}

\maketitle

%\newpage
%\tableofcontents
%\newpage

\section{Pauli-Spin-Paramagnetismus eines Elektronengases}

\subsection{Energie-Eigenwerte und großkanonische Zustandssumme}

\fehlt

\subsection{Tieftemperaturentwicklung des großkanonischen Potenzials}

\fehlt

\subsection{Teilchenzahlen}

\fehlt

\subsection{Tieftemperaturverhalten der Magnetisierung}

\fehlt

\subsection{Magnetische Suszeptibilität}

\fehlt

\subsection{Feste Teilchenzahl statt festes chemisches Potenzial}

\fehlt

\section{Landau-Diamagnetismus}

\subsection{Unabhängigkeit der Zustandssumme von Magnetfeld}

\fehlt

\subsection{Entartungsgrad von Landau-Niveaus}

\fehlt

\subsection{Zustandsdichte des Elektronengases im Magnetfeld}

\fehlt

\subsection{Großkanonisches Potenzial}

\fehlt

\subsection{Magnetisierung des Elektronengases}

\fehlt

\section{Bose-Einstein-Kondensation}

\subsection{Entartungsgrad und Zustandsdichte}

\fehlt

\subsection{Thermodynamischer Limes}

\fehlt

\subsection{BEC-Temperatur und Besetzungszahl des Grundzustandes}

\fehlt

\subsection{Mittlere quadratische Ausdehnung}

\fehlt

\subsection{Mittlere quadratische Ausdehnung bei Folgen der Boltzmann-Statistik}

\fehlt

\IfFileExists{\bibliographyfile}{
    \printbibliography
}{}

\end{document}

% vim: spell spelllang=de
