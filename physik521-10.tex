\input{header.tex}

%\subject{}
\title{physik521 – Übung 10}
%\subtitle{}
\author{
	Martin Ueding \\ \small{\href{mailto:mu@martin-ueding.de}{mu@martin-ueding.de}}
        \and Paul Manz
        \and Lino Lemmer \\ \small{\href{mailto:l2@uni-bonn.de}{l2@uni-bonn.de}}
}

\pagestyle{plain}

\begin{document}

\maketitle

%\newpage
%\tableofcontents
%\newpage

\section{Pauli-Spin-Paramagnetismus eines Elektronengases}

\subsection{Energie-Eigenwerte und großkanonische Zustandssumme}

\fehlt

\subsection{Tieftemperaturentwicklung des großkanonischen Potenzials}

\fehlt

\subsection{Teilchenzahlen}

\fehlt

\subsection{Tieftemperaturverhalten der Magnetisierung}

\fehlt

\subsection{Magnetische Suszeptibilität}

\fehlt

\subsection{Feste Teilchenzahl statt festes chemisches Potenzial}

\fehlt

\section{Landau-Diamagnetismus}

\subsection{Unabhängigkeit der Zustandssumme von Magnetfeld}

Es gibt eine feste Anzahl Teilchen und verschiedene Energien, daher muss ich
den kanonischen Formalismus benutzen.

Meine Zustände sind alle Impulse $\vec p \in \R^3$. Daher muss ich für die
Zustandssumme im Impulsraum, in dem $\hat{\vec p} = \vec p$ gilt, integrieren.
\begin{align*}
    Z_{\text C, 1}
    &= \int\limits_{\R^3} \dif{^3 p} \, \exp(-\beta \hat H(\vec p)) \\
    &= \int\limits_{\R^3} \dif{^3 p} \, \exp\del{-\beta \frac1{2m} \del{\vec p + \frac ec \vec A}^2} \\
    \intertext{%
        Jetzt ist zu zeigen, dass dies unabhängig von $\vec B$ ist. Wir werden
        sogar zeigen, dass dies unabhängig von $\vec A$ ist. Dazu führen wir
        eine Substitution durch:
        \[
            \vec q := \vec p + \frac ec \vec A.
        \]
        Diese Substitution ist für alle Impulsunabhängigen $\vec A$ definiert
        und es gilt vor allem $\dif{\vec q} = \dif{\vec p}$, da $\vec A$ nicht
        vom Impuls abhängt. Somit erhalten wir:
    }
    &= \int\limits_{\R^3} \dif{^3 q} \, \exp\del{-\beta \frac1{2m} \vec q^2} \\
    \intertext{%
        An dieser Stelle ist schon gezeigt, dass die Zustandssumme nicht von
        $\vec A$ abhängt. Jedoch können wir dieses Integral auch analytisch
        lösen. Wir wählen Kugelkoordinaten für den Impuls und können aufgrund
        der Isotropie den Winkelanteil schon integrieren. Es bleibt: 
    }
    &= 4 \piup \int_0^\infty \dif q \, q^2 \exp\del{-\beta \frac1{2m} q^2} \\
    \intertext{%
        Dieses Integral können wir mit partieller Integration lösen. Dabei ist
        die erste Stammfunktion zur Exponentialfunktion die Gauß'sche
        Fehlerfunktion. Die zweite Stammfunktion dazu ist beinhaltet wieder die
        Fehlerfunktion, jedoch haben wir dies nur mit Mathematica
        herausbekommen. Nach weiteren Rechnungen und weiteren Substitutionen
        erhalten wir:
    }
    &= \del{\frac{2 m \piup}\beta}^{3/2}.
\end{align*}

\subsection{Entartungsgrad von Landau-Niveaus}

Der Hamiltonoperator ist gegeben als:
\[
    \hat H = \frac1{2m} \del{\hat{\vec p} + \frac ec \vec A}^2
    = \frac1{2m} \del{\hat{\vec p} + \frac ec \hat x B \ev_y}^2,
\]
wobei $\ev_y$ den Einheitsvektor in $y$-Richtung bezeichnet.

Die Zeitentwicklung von $\vec p$ ist gegeben durch:
\[
    \dod{}t \hat{\vec p} = \frac1{\iup\hbar} [\hat{\vec p}, \hat H] + \dpd{}t \hat{\vec p}.
\]

Wir setzen den Hamiltonoperator ein und benutzen die Darstellung im Impulsraum,
so dass $\hat p = p$ und $\hat x = \iup\hbar \partial/\partial p_x$ ist. Der
Impulsoperator ist nicht explizit zeitabhängig, so dass wir nur den Kommutator
ausrechnen müssen, um die Zeitentwicklung zu erhalten.

\fehlt

\subsection{Zustandsdichte des Elektronengases im Magnetfeld}

\fehlt

\subsection{Großkanonisches Potenzial}

\fehlt

\subsection{Magnetisierung des Elektronengases}

\fehlt

\section{Bose-Einstein-Kondensation}

\subsection{Entartungsgrad und Zustandsdichte}

\fehlt

\subsection{Thermodynamischer Limes}

\fehlt

\subsection{BEC-Temperatur und Besetzungszahl des Grundzustandes}

\fehlt

\subsection{Mittlere quadratische Ausdehnung}

\fehlt

\subsection{Mittlere quadratische Ausdehnung bei Folgen der Boltzmann-Statistik}

\fehlt

\IfFileExists{\bibliographyfile}{
    \printbibliography
}{}

\end{document}

% vim: spell spelllang=de
