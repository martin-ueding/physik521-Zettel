\input{header.tex}

\setcounter{section}{0}
\renewcommand\thesection{H\,7.\arabic{section}}
\renewcommand\thesubsection{\thesection.\alph{subsection}}

\title{physik521: Übungsblatt 07}
\author{%
    Lino Lemmer \\ \small{\texttt{s6lilemm@uni-bonn.de}}
    \and
    Martin Ueding \\ \small{\texttt{mu@martin-ueding.de}}
    \and
    Paul Manz \\ \small{\texttt{p.m@uni-bonn.de}}
}

\begin{document}
\maketitle
\section{Zweiatomiges Molekül}

\section{Kühlung durch adiabatische Entmagnetisierung}

Hier ist nach der kanonischen Zustandssumme gefragt, daher betrachten wir dieses System auch im kanonischen Formalismus. Wir bestimmen $Z_\text C$:
\begin{align*}
    Z_\text C
    &= \prod_{i=1}^N \sum_{s_i = -1, 0, 1} \exp\del{\frac{\mu B s_i}{k T}} \\
    &= \del{\sum_{s_i = -1, 0, 1} \exp\del{\frac{\mu B s_i}{k T}}}^N \\
    &= \del{1 + 2 \cosh\del{\frac{\mu B s_i}{k T}}}^N.
\end{align*}

Daraus können wir die freie Energie $F(T) = - k T \ln(Z_\text C)$ bestimmen:
\[
    F(T) = - N k T \ln\del{1 + 2 \cosh\del{\frac{\mu B s_i}{k T}}}.
\]

Aus der freien Energie bestimmen wir die Entropie:
\begin{align*}
    S(T)
    &= - \dpd FT \\
    &= N k \del{
        \ln\del{1 + 2 \cosh\del{\frac{\mu B s_i}{k T}}}
        - \frac{\mu B}{kT} \frac{2 \sinh\del{\frac{\mu B s_i}{k T}}}{1 + 2\cosh\del{\frac{\mu B s_i}{k T}}}
    }.
\end{align*}

Dies stimmt mit dem Kontrollergebnis überein.

\section{Polymer-Modell (Gummi)}

\end{document}
