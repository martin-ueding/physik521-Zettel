\input{header.tex}

\setcounter{section}{0}
\renewcommand\thesection{H\,7.\arabic{section}}
\renewcommand\thesubsection{\thesection.\alph{subsection}}

\title{physik521: Übungsblatt 07}
\author{%
    Lino Lemmer \\ \small{\texttt{s6lilemm@uni-bonn.de}}
    \and
    Martin Ueding \\ \small{\texttt{mu@martin-ueding.de}}
    \and
    Paul Manz \\ \small{\texttt{p.m@uni-bonn.de}}
}

\begin{document}
\maketitle
\section{Zweiatomiges Molekül}

\section{Kühlung durch adiabatische Entmagnetisierung}

\subsection{Entropie}

Hier ist nach der kanonischen Zustandssumme gefragt, daher betrachten wir dieses System auch im kanonischen Formalismus. Wir bestimmen $Z_\text C$:
\begin{align*}
    Z_\text C
    &= \prod_{i=1}^N \sum_{s_i = -1, 0, 1} \exp\del{\frac{\mu B s_i}{k T}} \\
    &= \del{\sum_{s_i = -1, 0, 1} \exp\del{\frac{\mu B s_i}{k T}}}^N \\
    &= \del{1 + 2 \cosh\del{\frac{\mu B s_i}{k T}}}^N.
\end{align*}

Daraus können wir die freie Energie $F(T) = - k T \ln(Z_\text C)$ bestimmen:
\[
    F(T) = - N k T \ln\del{1 + 2 \cosh\del{\frac{\mu B s_i}{k T}}}.
\]

Aus der freien Energie bestimmen wir die Entropie:
\begin{align*}
    S(T)
    &= - \dpd FT \\
    &= N k \del{
        \ln\del{1 + 2 \cosh\del{\frac{\mu B s_i}{k T}}}
        - \frac{\mu B}{kT} \frac{2 \sinh\del{\frac{\mu B s_i}{k T}}}{1 + 2\cosh\del{\frac{\mu B s_i}{k T}}}
    }.
\end{align*}

Dies stimmt mit dem Kontrollergebnis überein.

\subsection{Abkühlung}

Die Temperatur wird auf $T_1$ fixiert. Wenn es adiabatisch geändert wird, ist
$\deltaup Q = 0$. Außerdem geht es so schnell, dass die Teilsysteme sich nicht
verändern können. $W_\text C$ bleibt also fest. Daher muss auch $Z_\text C$
sowie $F$ und $S$ konstant bleiben.

\subsection{Neue Temperatur}

Wir haben es nicht geschafft, analytisch zu zeigen, dass $S(T)$ injektiv ist.
Dies bedeutet, dass zwei verschiedene Entropien durch zwei verschiedene $B/T$
kommen muss. Als Anschauung haben wir zwei Plots erstellt, in denen Qualitativ
$S(B/T)$ und die Ableitung $S'(B/T)$ geplottet worden sind:

\includegraphics[width=.45\textwidth]{2b-S.pdf}
\hfill
\includegraphics[width=.45\textwidth]{2b-dS.pdf}

Es ist zu sehen, dass die Funktion streng monoton falled ist, die Ableitung ist
immer negativ. Daher ist die Funktion sogar bijektiv, also auch injektiv.

Es muss
\[
    \frac{B_1}{T_1} = \frac{B_2}{T_2}
\]
gelten, da $S$ konstant ist, $S$ injektiv ist und daher $\mu B/kT$ konstant
sein muss.

Man muss $B_2 < B_1$ wählen, damit es passt.

\subsection{Wärmekapazität}

Die Wärmekapazität ist laut Skript definiert als:
\[
    c_B(T) = \tdpd QTB = T \tdpd STB.
\]
Diese Ableitung bestimmen wir jetzt.

\begin{align*}
    c_B(T) 
    &= T \tdpd{}TB N k \del{
        \ln\del{1 + 2 \cosh\del{\frac{\mu B s_i}{k T}}}
        - \frac{\mu B}{kT} \frac{2 \sinh\del{\frac{\mu B s_i}{k T}}}{1 + 2\cosh\del{\frac{\mu B s_i}{k T}}}
    } \\
    \intertext{%
        Die beiden ersten Terme, die durch den Logarithmus und durch den Bruch
        vor dem Bruch entstehen, sind zusammen gerade 0. Es bleibt die
        Ableitung nach $T$ im zweiten Bruch, die sich mit der Quotientenregel
        gerade zu dem angegeben Zwischenergebnis umformen lässt.
    }
    &= Nk \del{\frac{\mu B}{kT}}^2 \frac{2 \cosh \cdot (1 + 2 \cosh) - 4 \sinh^2}{(1 + 2 \cosh)^2} \\
    &= Nk \del{\frac{\mu B}{kT}}^2 \frac{2 \cosh + 4}{(1 + 2 \cosh)^2}
\end{align*}

Die Energie, die das Spinsystem verliert, geht in das andere. Die Differenz ist:
\[
    \Deltaup Q = \int_{T_1}^{T_2} \dif T \, c_B(T).
\]
Im anderen System wird dies eine Temperaturänderung von $\Deltaup T = \Deltaup Q / c_V^p$ verursachen. Das Integral sieht schwer aus, allerdings dürfen wir den Integranden nähern zu:
\[
    c_B(T) \approx 4 Nk \del{\frac{\mu B}{kT}}^2.
\]

Somit wird das Integral lösbar und wir erhalten:
\[
    \Deltaup T = 4 N \frac{\mu^2 B^2}{k} \del{\frac 1{T_1} - \frac 1{T_2}}.
\]

\section{Polymer-Modell (Gummi)}

\end{document}
