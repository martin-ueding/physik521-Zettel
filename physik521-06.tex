\input{header.tex}

\setcounter{section}{0}
\renewcommand\thesection{H\,6.\arabic{section}}
\renewcommand\thesubsection{\thesection.\alph{subsection}}

\title{physik521: Übungsblatt 06}
\author{%
    Lino Lemmer \\ \small{\texttt{s6lilemm@uni-bonn.de}}
    \and
    Martin Ueding \\ \small{\texttt{mu@martin-ueding.de}}
    \and
    Paul Manz \\ \small{\texttt{p.m@uni-bonn.de}}
}

\begin{document}
\maketitle

\section{Energiefluktuationen im kanonischen Ensemble}

Zunächst gilt generell Folgendes:
\[
    \mw{(\Delta E)^2} = \mw{E^2} - \mw{E}^2.
\]

Wir bilden die Ableitung von $\mw E$ und fassen Mittelwertsbildungen zusammen:
\begin{align*}
    kT^2 \dpd{\mw E}T
    &= \frac{\sum_n \exp\del{-\frac{E_n}{kT}} E_n^2 \cdot Z_K - Z_K \cdot \mw E \sum_j \exp\del{-\frac{E_j}{kT}} E_j}{Z_K^2} \\
    &= \mw{E^2} - \mw{E}^2 \\
    &= \mw{(\Delta E)^2}
\end{align*}

Somit ist die Relation gezeigt.

Dies hängt mit der Wärmekapazität zusammen, da diese wie folgt definiert ist:
\[
    c_V = \tdpd ETV.
\]
Es unterscheidet sich also nur um einen Faktor $kT^2$.

Laut Skript sollen im Grenzfall $N \to \infty$ die Fluktuationen gegen 0 gehen.
Daraus soll dann folgen, dass der mikrokanonische und kanonische Formalismus
equivalent sind.

\end{document}
