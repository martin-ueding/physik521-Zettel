\input{header.tex}

%\subject{}
\title{physik521 Übung 8}
%\subtitle{}
\author{
    Lino Lemmer
    \and
    Paul Manz
    \and
    Martin Ueding \\ {\small \href{mailto:mu@martin-ueding.de}{mu@martin-ueding.de}}
}

\begin{document}

\maketitle

\newcommand\ZC{Z_\text C}

\section{System von unabhängigen harmonischen Ozsillatoren}

\subsection{Kanonische Zustandssumme}

Zuerst berechnen wir $Z_C$:
\begin{align*}
    Z_\text C
    &= \prod_{n=1}^N \sum_{n_i=0}^\infty \exp\del{- \frac{\hbar\omega \del{n_i + \frac 12}}{kT}} \\
    &= \prod_{n=1}^N \sum_{n_i=0}^\infty \exp\del{- \frac{\hbar\omega}{2kT}} \exp\del{- \frac{\hbar\omega n_i}{kT}} \\
    &= \prod_{n=1}^N \exp\del{- \frac{\hbar\omega}{2kT}} \sum_{n_i=1}^\infty \sbr{\exp\del{- \frac{\hbar\omega}{kT}}}^{n_i} \\
    &= \prod_{n=1}^N \exp\del{- \frac{\hbar\omega}{2kT}} \frac1{1 - \exp\del{- \frac{\hbar\omega}{kT}}} \\
    &= \del{\frac 12 \csch\del{\frac{\hbar\omega}{2kT}}}^N
\end{align*}

Die Energieeigenwerte eines Mikrozustandes $\ket n$ sind einfach die Summen der einzelnen Energieeigenwerte. Und die sind $\hbar\omega\cdot(n + 1/2)$. Oder ist mit $\ket n$ gemeint, dass alle $N$ Oszillatoren den Zustand $n$ haben? Dann ist die Energie einfach $N E_n$.

\subsection{Innere Energie}

Die innere Energie ist der Erwartungswert der Energie:
\begin{align*}
    U
    &= \braket E \\
    &= \sum_{n=0}^\infty W(n) E_n \\
    &= \frac1\ZC \sum_{n=0}^\infty \exp\del{-\frac{E_n}{kT}} E_n \\
    \intertext{%
        Dies schreiben wir jetzt als Ableitung von $\ZC$:
    }
    &= -\frac1\ZC \dpd{}{\frac1{kT}} \sum_{n=0}^\infty \exp\del{-\frac{E_n}{kT}} \\
    \intertext{%
        Das hinter der Ableitung ist gerade $\ZC$. Den Faktor $\frac1\ZC$ ganz
        vorne werden wir noch durch den natürlichen Logarithmus los.
    }
    &= -\dpd{}{\frac1{kT}} \ln(\ZC) \\
    \intertext{%
        Wir transformieren die partielle Ableitung noch mit
        \[
            \dpd{}T = \dpd{\frac1{kT}}T \dpd{}{\frac1{kT}} = - \frac1{kT^2} \dpd{}{\frac1{kT}}
            \implies
            \dpd{}{\frac1{kT}} = - kT^2 \dpd{}T
        \]
        und erhalten so:
    }
    &= kT^2 \dpd{}T \ln(\ZC) \\
    \intertext{%
        Nun setzen wir $\ZC$ ein und führen die Ableitung aus.
    }
    &= NkT^2 \frac{\frac 12 \frac{\hbar\omega}{2 k T^2} \csch\del{\frac{\hbar\omega}{2kT}}\coth\del{\frac{\hbar\omega}{2kT}}}{\frac 12 \csch\del{\frac{\hbar\omega}{2kT}}} \\
    &= N\frac{\hbar\omega}2 \coth\del{\frac{\hbar\omega}{2kT}}
\end{align*}

Wir berechnen den Erwartungswert für die Phononenzahl $n$:
\begin{align*}
    \bracket n
    &= \sum_n W(n) n \\
    &= \frac1\ZC \sum_n \exp\del{-\frac{\hbar \omega \del{n + \frac12}}{kT}} n \\
    \intertext{%
        Wir ziehen $\ZC$ unter alles.
    }
    &= \frac{\sum_n \exp\del{-\frac{\hbar \omega \del{n + \frac12}}{kT}} n}{\ZC} \\
    \intertext{%
        Wir schreiben $\ZC$ aus.
    }
    &= \frac{\sum_n \exp\del{-\frac{\hbar \omega \del{n + \frac12}}{kT}} n}{\sum_n \exp\del{-\frac{\hbar \omega \del{n + \frac12}}{kT}}} \\
    \intertext{%
        Nun addieren wir null.
    }
    &= \frac{\sum_n \exp\del{-\frac{\hbar \omega \del{n + \frac12}}{kT}} n}{\sum_n \exp\del{-\frac{\hbar \omega \del{n + \frac12}}{kT}}} + \frac12 - \frac12 \\
    \intertext{%
        Die $+\frac 12$ werden in den Zähler integriert.
    }
    &= \frac{\sum_n \exp\del{-\frac{\hbar \omega \del{n + \frac12}}{kT}} \del{n + \frac12}}{\sum_n \exp\del{-\frac{\hbar \omega \del{n + \frac12}}{kT}}} - \frac12 \\
    \intertext{%
        Wir wenden den Trick mit der Ableitung an und ziehen somit das $(n +
        \frac 12)$ in die Ableitung rein.
    }
    &= - \frac{\dpd{}{\frac{\hbar\omega}{kT}} \sum_n \exp\del{-\frac{\hbar \omega \del{n + \frac12} }{kT}}}{\sum_n \exp\del{-\frac{\hbar \omega \del{n + \frac12}}{kT}}} - \frac12 \\
    \intertext{%
        Das ganze können wir auch als Ableitung einer Logarithmusfunktion
        schreiben.
    }
    &= - \dpd{}{\frac{\hbar\omega}{kT}} \ln\del{\sum_n \exp\del{-\frac{\hbar \omega \del{n + \frac12}}{kT}}} - \frac12 \\
    \intertext{%
        Dies ist gerade $Z_{\text C,1}$.
    }
    &= -\dpd{}{\frac{\hbar\omega}{kT}} \ln(Z_{\text C,1}) - \frac12 \\
    \intertext{%
        Wir setzen unser vorheriges Ergebnis für $Z_{\text C,1}$ ein.
    }
    &= \dpd{}{\frac{\hbar\omega}{kT}} \ln\del{2 \sinh\del{\frac{\hbar\omega}{2kT}}} - \frac12 \\
    \intertext{%
        Die Ableitung führen wir aus:
    }
    &= \frac{2 \cosh(\ldots)}{2 \sinh(\ldots)} \frac 12 - \frac 12 \\
    \intertext{%
        Wir benutzen die Exponentialdarstellung der Funktionen.
    }
    &= \frac 12\del{\frac{\eup^{\ldots} + \eup^{-\ldots}}{\eup^{\ldots} - \eup^{-\ldots}} - \frac 12} \\
    \intertext{%
        Nach etwas umstellen erhalten wir:
    }
    &= \frac1{\exp\del{\frac{\hbar\omega}{kT}} - 1}
\end{align*}

Dies ist der gesuchte Erwartungswert.

Wir können nun das Zwischenergebnis benutzen, um die innere Energie noch etwas kompakter auszudrücken. Dieses Ergebnis ist:
\[
    \braket n = \frac12 \coth\del{\frac{\hbar\omega}{2kT}} - \frac 12
\]

Unsere bisherige Form für die innere Energie ist:
\begin{equation}
    \label{eq:U}
    U = N \frac{\hbar\omega}2 \coth\del{\frac{\hbar\omega}{2kT}}
\end{equation}

Dies können wir jetzt benutzen, um zu schreiben:
\[
    U = N \hbar \omega \del{\bracket n + \frac12}
\]

Wir benutzen die Form \eqref{eq:U}, um die Abhängigkeit der inneren Energie von $kT/\hbar\omega$ zu plotten. Dies ist in Abbildung~\ref{fig:U} dargestellt.

\begin{figure}[htbp]
    \centering
    \includegraphics[width=.6\linewidth]{2b.png}
    \caption{%
        Abhängigkeit der inneren Energie von $kT/\hbar\omega$.
    }
    \label{fig:U}
\end{figure}

\subsection{Freie Energie}

Die freie Energie errechnen wir aus der Zustandssumme:
\begin{align*}
    F
    &= - k T \ln(\ZC) \\
    &= - N k T \ln\del{\frac 12 \csch\del{\frac{\hbar\omega}{2kT}}} \\
    &= N k T \ln\del{2 \sinh\del{\frac{\hbar\omega}{2kT}}} \\
    \intertext{%
        Im Skript ist jetzt noch mit Formel (4.77) eine weitere Umformung
        gemacht:
    }
    &= N \del{kT \ln\del{1 - \exp\del{-\frac{\hbar\omega}{kT}}} + \frac{\hbar\omega}2}
\end{align*}

\subsection{Ausdruck für die Wärmekapazität}

In Abbildung~\ref{fig:1d-c} ist dieses Verhalten dargestellt.

\begin{figure}[htbp]
    \centering
    \includegraphics[width=0.6\linewidth]{1d.png}
    \caption{%
        Wärmekapazität $c$ in Einheiten von $k$ gegen $\hbar\omega / kT$. Die
        violette Linie ist der erste Summand, die Blaue der Zweite und die
        Gelbe die Summe der beiden.
    }
    \label{fig:1d-c}
\end{figure}

\section{Gibb'sches Paradoxon}

\subsection{Druck- und Temperaturausgleich}

Gegeben sind:
\[
    N_1
    \eqnsep
    N_2
    \eqnsep
    p_1
    \eqnsep
    p_2
    \eqnsep
    T_1
    \eqnsep
    T_2
\]

\subsection{Änderung der Entropie}

\section{Maxwell'sche Geschwindigkeitsverteilung}


\IfFileExists{\bibliographyfile}{
    \printbibliography
}{}

\end{document}

% vim: spell spelllang=de
