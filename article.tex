\input{header.tex}

%\subject{}
\title{physik521 Übung 8}
%\subtitle{}
\author{
    Lino Lemmer
    \and
    Paul Manz
    \and
    Martin Ueding \\ {\small \href{mailto:mu@martin-ueding.de}{mu@martin-ueding.de}}
}

\begin{document}

\maketitle

\section{System von unabhängigen harmonischen Ozsillatoren}

\subsection{Kanonische Zustandssumme}

Zuerst berechnen wir $Z_C$:
\begin{align*}
    Z_\text C
    &= \sum_{n_i=1}^\infty \exp\del{- \frac{\hbar\omega \del{n_i + \frac 12}}{kT}} \\
    &= \sum_{n_i=1}^\infty \exp\del{- \frac{\hbar\omega}{2kT}} \exp\del{- \frac{\hbar\omega n_i}{kT}} \\
    &= \exp\del{- \frac{\hbar\omega}{2kT}} \sum_{n_i=1}^\infty \sbr{\exp\del{- \frac{\hbar\omega}{kT}}}^{n_i} \\
    &= \exp\del{- \frac{\hbar\omega}{2kT}} \frac1{1 - \exp\del{- \frac{\hbar\omega}{kT}}} \\
    &= \frac 12 \csch\del{- \frac{\hbar\omega}{kT}} \\
\end{align*}

\subsection{Innere Energie}

\subsection{Freie Energie}

\subsection{Ausdruck für die Wärmekapazität}

In Abbildung~\ref{fig:1d-c} ist dieses Verhalten dargestellt.

\begin{figure}[htbp]
    \centering
    \includegraphics[width=0.6\linewidth]{1d.png}
    \caption{%
        Wärmekapazität $c$ in Einheiten von $k$ gegen $\hbar\omega / kT$. Die
        violette Linie ist der erste Summand, die Blaue der Zweite und die
        Gelbe die Summe der beiden.
    }
    \label{fig:1d-c}
\end{figure}

\section{Gibb'sches Paradoxon}

\subsection{Druck- und Temperaturausgleich}

Gegeben sind:
\[
    N_1
    \eqnsep
    N_2
    \eqnsep
    p_1
    \eqnsep
    p_2
    \eqnsep
    T_1
    \eqnsep
    T_2
\]

\subsection{Änderung der Entropie}

\section{Maxwell'sche Geschwindigkeitsverteilung}


\IfFileExists{\bibliographyfile}{
    \printbibliography
}{}

\end{document}

% vim: spell spelllang=de
