\input{header.tex}

\hypersetup{
	pdftitle=
}

%\subject{}
\title{physik521 -- Übung 9}
%\subtitle{}
\author{
	Martin Ueding 
        \and
        Paul Manz
        \and
        Lino Lemmer
}

\begin{document}

\maketitle

\section{Das ideale Fermi-Gas}

\newcommand\kB{k_\text B}
\newcommand\ZG{Z_\text G}
\newcommand\ZGe{Z_{\text G, 1}}
\newcommand\Eai{E_{\alpha_i}}
\newcommand\isum{\sum_{i=1}^\infty }

Wir übernehmen folgende Formeln aus dem Skript:
\begin{gather*}
    \ZG = \prod_{i=1}^\infty \del{1 + \exp\del{- \frac{\Eai - \mu}{\kB T}}} \\
    \Omega = - \kB T \isum \ln \del{\ZGe(\alpha_i, \mu, T)} \\
    \rho(E) = \isum \delta(E - \Eai)
\end{gather*}

Alle diese Aufgaben stehen so im Skript vorgerechnet. Wir haben uns bei
unseren Rechnungen an das Skript gehalten, jedoch wieder versucht jeden
Schritt hier zur erklären.

\subsection{Allgemeine Ausdrücke}

\subsubsection{Mittlere Teilchenzahl}

Die Funktion $f(\Eai)$ gibt die Anzahl der Besetzungen des Zustandes $\alpha_i$ an. Somit wie im Skript:
\[
    \bracket N = \isum f(\Eai).
\]

Wenn man jetzt $\rho$ dazunimmt, geht dies so:
\[
    \bracket N = \int \dif E \, \rho(E) f(E).
\]

Dies funktioniert, da $\rho$ im Integral über $E$ an jedem Energiewert $E$ die
Anzahl der Zustände $\alpha_i$, die diese Energie haben, gibt. Somit wird die
Vielfachheit der Energiezustände von der Summe in das $\rho$ verlagert.

\subsubsection{Entropie}

Mit $\Omega = U - TS - \mu N$ kann man die Entropie $S$ als Ableitung schreiben:
\[
    S = - \tdpd\Omega T{\mu, V}.
\]

Diese rechnen wir jetzt konkret aus:
\begin{align*}
    S &= \kB \isum \ln(\ldots) + \kB T \isum \frac{\exp(\ldots)}{1 + \exp(\ldots)} \frac{\Eai - \mu}{\kB T^2} \\
      &= - \frac\Omega T + \isum f(\Eai) \frac{\Eai - \mu}T.
\end{align*}

\subsubsection{Innere Energie}

Mit der am Anfang der vorherigen Teilaufgabe genannten Relation können wir die innere Energie als $U = \Omega + TS + \mu N$ schreiben. Wir rechnen weiter:
\begin{align*}
    U &= \Omega + TS + \mu N \\
      &= \Omega - \Omega + \isum f(\Eai) \cdot (\Eai - \mu) + \mu N \\
      &= \isum f(\Eai) \cdot \Eai + \isum f(\Eai) \cdot \Eai + \mu N \\
      &= \bracket E - \bracket N \mu + \mu N. \\
    \intertext{%
        Wenn man das Limit $N \to \infty$ annimmt, kann man $\lim_{N\to\infty}
        \bracket N = N$ annehmen und weiter vereinfachen:
    }
      &= \bracket E.
\end{align*}

\subsubsection{Freie Energie}

Die freie Energie $F$ ist wieder ein transformiertes Potential, wir benutzen
hier $F = \Omega + \mu N$. Dabei sehen wir jedoch nicht, wie sich dies weiter
vereinfacht.

\subsection{Druck des Gases}

\subsubsection{Bei festem chemischen Potential}

\fehlt

\subsubsection{Bei fester Teilchenzahl}

\fehlt

\subsubsection{Diskussion}

\fehlt

\subsection{Spezifische Wärme}

Die Definition der spezifischen Wärme ist:
\[
    c_V = \tdpd QTV.
\]

Hier ist die Variante mit
\[
    c_V = - T \tdpd ST{V,\mu}
\]
nützlich.

\subsection{Tieftemperaturentwicklung}

\fehlt

\IfFileExists{\bibliographyfile}{
    \printbibliography
}{}

\end{document}

% vim: spell spelllang=de
