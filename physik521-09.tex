\input{header.tex}

\hypersetup{
	pdftitle=
}

%\subject{}
\title{physik521 -- Übung 9}
%\subtitle{}
\author{
	Martin Ueding 
        \and
        Paul Manz
        \and
        Lino Lemmer
}

\begin{document}

\maketitle

\section{Das ideale Fermi-Gas}

\newcommand\kB{k_\text B}
\newcommand\ZG{Z_\text G}
\newcommand\ZGe{Z_{\text G, 1}}
\newcommand\Eai{E_{\alpha_i}}
\newcommand\isum{\sum_{i=1}^\infty }

Wir übernehmen folgende Formel aus dem Skript:
\begin{gather*}
    \ZG = \prod_{i=1}^\infty \del{1 + \exp\del{- \frac{\Eai - \mu}{\kB T}}} \\
    \Omega = - \kB T \isum \ln \del{\ZGe(\alpha_i, \mu, T)} \\
    \rho(E) = \isum \delta(E - \Eai)
\end{gather*}

\subsection{Allgemeine Ausdrücke}

\subsubsection{Mittlere Teilchenzahl}

Die Funktion $f(\Eai)$ gibt die Anzahl der Besetzungen des Zustandes $\alpha_i$ an. Somit wie im Skript:
\[
    \bracket N = \isum f(\Eai).
\]

Wenn man jetzt $\rho$ dazunimmt, geht dies so:
\[
    \bracket N = \int \dif E \, \rho(E) f(E).
\]

Dies funktioniert, da $\rho$ im Integral über $E$ an jedem Energiewert $E$ die
Anzahl der Zustände $\alpha_i$, die diese Energie haben, gibt. Somit wird die
Multiplizität der Energiezustände von der Summe in das $\rho$ verlagert.

\IfFileExists{\bibliographyfile}{
    \printbibliography
}{}

\end{document}

% vim: spell spelllang=de
