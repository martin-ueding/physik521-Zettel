\input{header.tex}

\setcounter{section}{1}
\renewcommand\thesection{H\,3.\arabic{section}}
\renewcommand\thesubsection{\thesection.\alph{subsection}}

\title{physik521: Übungsblatt 03}
\author{%
    Lino Lemmer \\ \small{\texttt{s6lilemm@uni-bonn.de}}
    \and
    Martin Ueding \\ \small{\texttt{mu@martin-ueding.de}}
    \and
    Paul Manz \\ \small{\texttt{p.m@uni-bonn.de}}
}

\begin{document}
\maketitle

\section{Joule-Thomson-Effekt}
\subsection{Enthalpie und Irreversiblität}
Stelle zunächst fest, dass Teilchenzahl und Wärmeenergie unverändert bleiben, da wir einen adiabatischen Prozess in einem abgeschlossenen System betrachten.
\begin{align*}
\dif N &= 0 \\
\delta Q &= 0 \\
\end{align*}
Der erste Hauptsatz führt dann auf:
\[ \Delta U = U_2 - U_1 = \Delta W = -\int_{V_1}^{0} p_1 \dif V  + \int_{0}^{V_2} p_2 \dif V= p_1 V_1 - p_2 V_2 \]
Für die Enthalpie gilt also:
\begin{align*}
H &= U+pV \\
\implies \Delta H &= U_1 + p_1 V_1 - (U_2 + p_2 V_2) =0 
\end{align*}  
Aus der Definition der Enthalpie folgt:
\begin{align*}
\dif H &= T \dif S + V \dif P + \mu \dif N = 0 \\
\iff T \dif S &= V \dif P > 0 = \delta Q \\
\end{align*}
Der Prozess läuft also irreversibel ab.

\subsection{Joule-Thomson-Koeffizient}


\end{document}
