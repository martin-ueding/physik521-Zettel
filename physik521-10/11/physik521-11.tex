% Für Seitenformatierung

\documentclass[DIV=15]{scrartcl}

% Zeilenumbrüche

\parindent 0pt
\parskip 6pt

% Für deutsche Buchstaben und Synthax

\usepackage[ngerman]{babel}

% Für Auflistung mit speziellen Aufzählungszeichen

\usepackage{paralist}

% zB für \del, \dif und andere Mathebefehle

\usepackage{amsmath}
\usepackage{commath}
\usepackage{amssymb}

% für nicht kursive griechische Buchstaben

\usepackage{txfonts}

% Für \SIunit[]{} und \num in deutschem Stil

\usepackage[output-decimal-marker={,}]{siunitx}
\usepackage[utf8]{inputenc}

% Für \sfrac{}{}, also inline-frac

\usepackage{xfrac}

% Für Einbinden von pdf-Grafiken

\usepackage{graphicx}

% Umfließen von Bildern

\usepackage{floatflt}

% Für Links nach außen und innerhalb des Dokumentes

\usepackage{hyperref}

% Für weitere Farben

\usepackage{color}

% Für Streichen von z.B. $\rightarrow$

\usepackage{centernot}

% Für Befehl \cancel{}

\usepackage{cancel}

% Für Layout von Links

\hypersetup{
	citecolor=black,
	colorlinks=true,
	linkcolor=black,
	urlcolor=blue,
}

% Verschiedene Mathematik-Hilfen

\newcommand \e[1]{\cdot10^{#1}}
\newcommand\p{\partial}

\newcommand\half{\frac 12}
\newcommand\shalf{\sfrac12}

\newcommand\skp[2]{\left\langle#1,#2\right\rangle}
\newcommand\mw[1]{\left\langle#1\right\rangle}
\renewcommand \exp[1]{\mathrm e^{#1}}

% Nabla und Kombinationen von Nabla

\renewcommand\div[1]{\skp{\nabla}{#1}}
\newcommand\rot{\nabla\times}
\newcommand\grad[1]{\nabla#1}
\newcommand\laplace{\triangle}
\newcommand\dalambert{\mathop{{}\Box}\nolimits}

%Für komplexe Zahlen

\renewcommand \i{\mathrm i}
\renewcommand{\Im}{\mathop{{}\mathrm{Im}}\nolimits}
\renewcommand{\Re}{\mathop{{}\mathrm{Re}}\nolimits}

%Für Bra-Ket-Notation

\newcommand\bra[1]{\left\langle#1\right|}
\newcommand\ket[1]{\left|#1\right\rangle}
\newcommand\braket[2]{\left\langle#1\left.\vphantom{#1 #2}\right|#2\right\rangle}
\newcommand\braopket[3]{\left\langle#1\left.\vphantom{#1 #2 #3}\right|#2\left.\vphantom{#1 #2 #3}\right|#3\right\rangle}

\newcommand{\eqnsep}{,\quad}


%\subject{}
\title{physik521 – Übung 11}
%\subtitle{}
\author{
	Martin Ueding \\ \small{\href{mailto:mu@martin-ueding.de}{mu@martin-ueding.de}}
        \and Paul Manz
        \and Lino Lemmer \\ \small{\href{mailto:l2@uni-bonn.de}{l2@uni-bonn.de}}
}

\pagestyle{plain}

\newcommand\kB{k_\text B}
\newcommand\muB{\mu_\text B}

\begin{document}

\maketitle

\section{}

\fehlt
\section{}


\fehlt
\section{Heisenberg-Modell und reduzierte Dichtematrix}
\subsection{}
Gegeben ist der Hamiltonian
\[H = J \ \vec{S}^1 \cdot \vec{S}^2 \hspace{1.5cm} J>0 \]
Es gilt zu zeigen, dass:
\[H= \frac{J}{2}(\vec{S})^2 + C = \frac{J}{4}(S_+S_-+S_-S_++2(S_3)^2)+C\]

Wir beginnen mit der ersten Gleichung:
\begin{align*}
\frac{J}{2}(\vec{S}^2) &= \frac{J}{2} \sum_i (S^1_i + S^2_i)^2 \\
&= \frac{J}{2} \del{2S^1_1S^2_1+2S^1_2S^2_2+S^1_3S^2_3 + \sum_{i,j} (S^j_i)^2} \\
&= \frac{J}{2} \del{2 \ \vec{S}^1 \cdot \vec{S}^2 + \sum_{i,j} (S^j_i)^2}  \\
(S^j_i)^2 &= \frac{1}{4} \mathbb{1} \\
\implies J \ \vec{S}^1 \cdot \vec{S}^2 &= \frac{J}{2}(\vec{S})^2 + C
\end{align*}
Die zweite Gleichung:
\begin{align*}
S_+S_- &= (S_1 + iS_2)(S_1 - iS_2)  \\
&= (S_1)^2 + (S_2)^2 - iS_1S_2 + iS_2S_1 \\
&= (S_1)^2 + (S_2)^2 - i[S_1,S_2] \\
&= (S_1)^2 + (S_2)^2 + S_3 \\
\end{align*}
Analog finden wir: 
\[ S_-S_+ = (S_1)^2 + (S_2)^2 - S_3 \]
Es folgt:
\begin{align*}
\frac{J}{4}(S_+S_-+S_-S_++2(S_3)^2) &= \frac{J}{4}(2(S_1)^2+2(S_2)^2+2(S_3)^2) \\
&= \frac{J}{2}(\vec{S})^2
\end{align*}
Die Eigenzustände dieses Hamiltonian lassen sich durch die Einteilchenspinoren mithilfe der Clebsch-Gordan-Koeffizienten darstellen. Wir erhalten:
\begin{align*}
\ket{S=1, S_3=+1} &= \ket{\uparrow} \otimes \ket{\uparrow} \\
\ket{1,0} &= \frac{1}{\sqrt{2}}\ket{\uparrow} \otimes \ket{\downarrow} + \frac{1}{\sqrt{2}} \ket{\downarrow} \otimes \ket{\uparrow} \\
\ket{1,-1} &= \ket{\downarrow} \otimes \ket{\downarrow} \\
\ket{0,0} &= \frac{1}{\sqrt{2}} \ket{\uparrow} \otimes \ket{\downarrow} - \frac{1}{\sqrt{2}} \ket{\downarrow} \otimes \ket{\uparrow}
\end{align*}

Der Singulettzustand hat einen Energieeigenwert von $E_0=0$ die Triplettzustände haben alle den Eigenwert $E_1=1(1+1)\frac{J}{2}=J$. Wir können die Zustände als kanonisches Ensemble auffassen, da die Teilchenzahl fest ist. Dann gilt für die Wahrscheinlichkeitsverteilung:
\[h_i=\frac{1}{Z_c} \mathrm{e}^{-\beta E_i} \]
mit \[Z_c=1+ 3\mathrm{e}^{-\beta J} \].

Berechne in dieser Basis den Dichteoperator:
\begin{align*}
W_c &:= \sum_i h_i \ket{\psi_i}\bra{\psi_i} \\
&= \frac{1}{Z_c} \del{\ket{0,0}\bra{0,0}+\mathrm{e}^{-\beta J} \sum_{S_z} \ket{1,S_z}\bra{1,S_z}} 
\end{align*}


\subsection{}
Berechne die reduzierte Dichtematrix
\begin{align*}
W_S &= \sum_{S^2_z} \braket{S^2,S^2_z | W_c | S^2,S^2_z} \\
&= \braket{\uparrow_R | W_c | \uparrow_R} + \braket{\downarrow_R | W_c | \downarrow_R}  \\
\braket{\uparrow_R | W_c | \uparrow_R} &= \frac{1}{Z_c} \del{ \braket{\uparrow_R | 0,0}\braket{0,0 | \uparrow_R}} \\
&+ \frac{\mathrm{e}^{-\beta J}}{Z_C} \del{ \braket{\uparrow_R | 1,-1}\braket{1,-1 | \uparrow_R} + \braket{\uparrow_R | 1,0}\braket{1,0 | \uparrow_R} + \braket{\uparrow_R | 1,+1}\braket{1,+1 | \uparrow_R} } \\
&=\frac{1}{Z_c}\del{\frac{1}{2}\ket{\downarrow_S} \bra{\downarrow_S} +  \mathrm{e}^{-\beta J} \del{ \ket{\uparrow_S}\bra{\uparrow_S} + \frac{1}{2} \ket{\downarrow_S}\bra{\downarrow_S}}} \\
\braket{\uparrow_R | W_c | \uparrow_R} &= \frac{1}{Z_c}\del{\frac{1}{2}\ket{\uparrow_S} \bra{\uparrow_S} +  \mathrm{e}^{-\beta J} \del{ \ket{\downarrow_S}\bra{\downarrow_S} + \frac{1}{2} \ket{\uparrow_S}\bra{\uparrow_S}}} \\
\implies W_S &= \frac{1}{Z_C} \del{ \frac{1}{2}\ket{\downarrow_S} \bra{\downarrow_S} + \frac{1}{2}\ket{\uparrow_S} \bra{\uparrow_S} + \mathrm{e}^{-\beta J}\del{\ket{ \frac{3}{2} \downarrow_S}\bra{\downarrow_S} + \frac{3}{2} \ket{\uparrow_S}\bra{\uparrow_S}}}
\end{align*}


\subsection{}
Für $T \longrightarrow 0$ verschwindet $\mathrm{e}^{-\beta J}$. Aus der Dichtematrix wird dann:
\[W_C= \ket{0,0}\bra{0,0} \] also der Projektor auf den Grundzustand.
Aus der reduzierten Dichtematrix wird:
\[W_S = \frac{1}{2}\ket{\downarrow_S} \bra{\downarrow_S} + \frac{1}{2}\ket{\uparrow_S} \bra{\uparrow_S} \]
Im ersten Fall ist als Energiezustand nur der Grundzustand möglich. Für den Entartungsgrad gilt dann $\Omega(T\rightarrow 0)=1$ für die Entropie also \[S=k_{\text B} \ln \Omega =0 \]
Im Zweiten Fall gilt:
\begin{align*}
\Omega(T\rightarrow 0) &= 2 \\
S &=k_{\text B} \ln 2
\end{align*}


\IfFileExists{\bibliographyfile}{
    \printbibliography
}{}

\end{document}

% vim: spell spelllang=de tw=79
