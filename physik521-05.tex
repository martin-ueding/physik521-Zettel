\input{header.tex}

\setcounter{section}{0}
\renewcommand\thesection{H\,5.\arabic{section}}
\renewcommand\thesubsection{\thesection.\alph{subsection}}

\title{physik521: Übungsblatt 05}
\author{%
    Lino Lemmer \\ \small{\texttt{s6lilemm@uni-bonn.de}}
    \and
    Martin Ueding \\ \small{\texttt{mu@martin-ueding.de}}
    \and
    Paul Manz \\ \small{\texttt{p.m@uni-bonn.de}}
}

\begin{document}
\maketitle
\section{Sattelpunktmethode}
\subsection{}
Gegeben ist das Integral
\begin{align}
    I &= \lim_{N\to\infty}\int_a^b\!\dif x\; \exp{Nf(x)} \label{eq:integral}
    \intertext{$f$ hat ein globales Maximum bei $x=x_0$. Um diesen Punkt
    Taylorn wir die Funktion:}
    f(x) &= f\del{x_0} + f'\del{x_0}\del{x-x_0} +
    \frac{f''\del{x_0}}2\del{x-x_0}^2 + \dots
    \label{eq:taylor}
    \intertext{%
        Setzen wir \eqref{eq:taylor} in \eqref{eq:integral} ein und
        brechen nach dem dritten Glied ab, erhalten wir
    }
    I &= \lim_{N\to\infty}\int_a^b\!\dif x\;
    \exp{Nf\del{x_0} + Nf'\del{x_0}x + \frac{N}2f''
    \del{x_0}\del{x-x_0}^2}\notag
    \intertext{%
        Die erste Ableitung verschwindet bei $x_0$, da dort das Maximum der
        Funktion ist.
    }
    &= \lim_{N\to\infty}\int_a^b\!\dif x\;
    \exp{Nf\del{x_0}}\exp{\frac{N}2f''\del{x_0}\del{x-x_0}^2}\notag \\
    &= \lim_{N\to\infty}\exp{Nf\del{x_0}}\int_a^b\!\dif x\;
    \exp{\frac{N}2f''\del{x_0}\del{x-x_0}^2}\notag \\
    &= \lim_{N\to\infty}\exp{Nf\del{x_0}}\int_a^b\!\dif x\;
    \exp{\frac N2f''\del{x_0}x^2}\exp{Nf''\del{x_0}x_0x}
    \exp{\frac N2f''\del{x_0}x_0^2} \notag \\
    &= \lim_{N\to\infty}\exp{Nf\del{x_0}}\exp{\frac N2f''\del{x_0}x_0^2}
    \int_a^b\!\dif x\;
    \exp{\frac N2f''\del{x_0}x^2}\exp{Nf''\del{x_0}x_0x} \notag
    \intertext{mit $a = -\frac N2 f''\del{x_0}$:}
    &= \lim_{N\to\infty} \exp{Nf\del{x_0}} \exp{-ax_0^2} \int_a^b\!\dif x\;
    \exp{2ax_0x}\exp{-ax^2} \notag
    \intertext{mit $\ii\omega = -2ax_0$}
    &= \lim_{N\to\infty} \exp{Nf\del{x_0}} \exp{\frac{\omega^2}{4a}} \int_a^b
    \!\dif x\; \exp{-\ii\omega x}\exp{-ax^2} \label{eq:integral_getaylort}
    \intertext{%
        Nun betrachten wir die Formel für Gaußsche Integrale
    }
    \sqrt{\frac{\piup}a}\exp{-\frac{\omega^2}{4a}}&=
    \int\!\dif t\;\exp{-\ii\omega t}\exp{-at^2}\label{eq:Gauss} 
    \intertext{%
        Lösen wir nun \eqref{eq:integral_getaylort} mit \eqref{eq:Gauss}
        auf, erhalten wir:
    }
    &= \lim_{N_\to\infty} \exp{Nf\del{x_0}} \exp{\frac{\omega^2}{4a}}
    \sqrt{\frac\piup a}\exp{-\frac{\omega^2}{4a}}\notag \\
    &= \lim_{N_\to\infty} \exp{Nf\del{x_0}} \sqrt{\frac\piup a}\notag \\
    &= \lim_{N_\to\infty} \exp{Nf\del{x_0}}
    \sqrt{-\frac{2\piup}{Nf''\del{x_0}}}\notag
    \intertext{%
        Da die zweite Ableitung an der Maximalstelle negativ ist, folgt
    }
    &= \lim_{N_\to\infty} \exp{Nf\del{x_0}}
    \sqrt{\frac{2\piup}{N\abs{f''\del{x_0}}}}\notag
\end{align}
\subsection{}

\section{Ensemble quantenmechanischer harmonischer Oszillatoren}
\subsection{}
\subsection{}
\subsection{}
\subsection{}
\subsection{}
\end{document}
