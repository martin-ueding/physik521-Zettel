\input{header.tex}

\setcounter{section}{0}
\renewcommand\thesection{H\,5.\arabic{section}}
\renewcommand\thesubsection{\thesection.\alph{subsection}}

\title{physik521: Übungsblatt 05}
\author{%
    Lino Lemmer \\ \small{\texttt{s6lilemm@uni-bonn.de}}
    \and
    Martin Ueding \\ \small{\texttt{mu@martin-ueding.de}}
    \and
    Paul Manz \\ \small{\texttt{p.m@uni-bonn.de}}
}

\begin{document}
\maketitle
\section{Sattelpunktmethode}
\subsection{}
Gegeben ist das Integral
\begin{align}
    I &= \lim_{N\to\infty}\int_a^b\!\dif x\; \exp{Nf(x)} \label{eq:Integral}
    \intertext{$f$ hat ein globales Maximum bei $x=x_0$. Um diesen Punkt
    Taylorn wir die Funktion:}
    f(x) &= f\del{x_0} + f'\del{x_0}x + \frac{f''\del{x_0}}2x^2 + \dots
    \label{eq:Taylor}
    \intertext{%
        Setzen wir \eqref{eq:Taylor} in \eqref{eq:Integral} ein und
        brechen nach dem dritten Glied ab, erhalten wir
    }
    I &= \lim_{N\to\infty}\int_a^b\!\dif x\;
    \exp{Nf\del{x_0} + Nf'\del{x_0}x + \frac{N}2f''\del{x_0}x^2}\notag
    \intertext{%
        Die erste Ableitung verschwindet bei $x_0$, da dort das Maximum der
        Funktion ist.
    }
    &= \lim_{N\to\infty}\int_a^b\!\dif x\;
    \exp{Nf\del{x_0}}\exp{\frac{N}2f''\del{x_0}x^2}\notag \\
    &= \lim_{N\to\infty}\exp{Nf\del{x_0}}\int_a^b\!\dif x\;
    \exp{\frac{N}2f''\del{x_0}x^2}\label{eq:Integral_getaylort}
    \intertext{%
        Nun betrachten wir die Formel für Gaußsche Integrale
    }
    \int\!\dif t\;\exp{-\ii\omega t}\exp{-at^2} &=
    \sqrt{\frac{\piup}a\exp{-\frac{\omega^2}{4a}}\label{eq:Gauß}
\end{align}
\subsection{}

\section{Ensemble quantenmechanischer harmonischer Oszillatoren}
\subsection{}
\subsection{}
\subsection{}
\subsection{}
\subsection{}
\end{document}
