% Für Seitenformatierung

\documentclass[DIV=15]{scrartcl}

% Zeilenumbrüche

\parindent 0pt
\parskip 6pt

% Für deutsche Buchstaben und Synthax

\usepackage[ngerman]{babel}

% Für Auflistung mit speziellen Aufzählungszeichen

\usepackage{paralist}

% zB für \del, \dif und andere Mathebefehle

\usepackage{amsmath}
\usepackage{commath}
\usepackage{amssymb}

% für nicht kursive griechische Buchstaben

\usepackage{txfonts}

% Für \SIunit[]{} und \num in deutschem Stil

\usepackage[output-decimal-marker={,}]{siunitx}
\usepackage[utf8]{inputenc}

% Für \sfrac{}{}, also inline-frac

\usepackage{xfrac}

% Für Einbinden von pdf-Grafiken

\usepackage{graphicx}

% Umfließen von Bildern

\usepackage{floatflt}

% Für Links nach außen und innerhalb des Dokumentes

\usepackage{hyperref}

% Für weitere Farben

\usepackage{color}

% Für Streichen von z.B. $\rightarrow$

\usepackage{centernot}

% Für Befehl \cancel{}

\usepackage{cancel}

% Für Layout von Links

\hypersetup{
	citecolor=black,
	colorlinks=true,
	linkcolor=black,
	urlcolor=blue,
}

% Verschiedene Mathematik-Hilfen

\newcommand \e[1]{\cdot10^{#1}}
\newcommand\p{\partial}

\newcommand\half{\frac 12}
\newcommand\shalf{\sfrac12}

\newcommand\skp[2]{\left\langle#1,#2\right\rangle}
\newcommand\mw[1]{\left\langle#1\right\rangle}
\renewcommand \exp[1]{\mathrm e^{#1}}

% Nabla und Kombinationen von Nabla

\renewcommand\div[1]{\skp{\nabla}{#1}}
\newcommand\rot{\nabla\times}
\newcommand\grad[1]{\nabla#1}
\newcommand\laplace{\triangle}
\newcommand\dalambert{\mathop{{}\Box}\nolimits}

%Für komplexe Zahlen

\renewcommand \i{\mathrm i}
\renewcommand{\Im}{\mathop{{}\mathrm{Im}}\nolimits}
\renewcommand{\Re}{\mathop{{}\mathrm{Re}}\nolimits}

%Für Bra-Ket-Notation

\newcommand\bra[1]{\left\langle#1\right|}
\newcommand\ket[1]{\left|#1\right\rangle}
\newcommand\braket[2]{\left\langle#1\left.\vphantom{#1 #2}\right|#2\right\rangle}
\newcommand\braopket[3]{\left\langle#1\left.\vphantom{#1 #2 #3}\right|#2\left.\vphantom{#1 #2 #3}\right|#3\right\rangle}

\newcommand{\eqnsep}{,\quad}


\usepackage[
    backend=bibtex,
    style=authoryear-icomp,
    hyperref=true
]{biblatex}

\addbibresource{central-bibtex/Central.bib}

\newcommand\eup{\mathrm e}
\newcommand\iup{\mathrm i}

\setcounter{section}{0}
\renewcommand\thesection{H\,5.\arabic{section}}
\renewcommand\thesubsection{\thesection.\alph{subsection}}

\title{physik521: Übungsblatt 05}
\author{%
    Lino Lemmer \\ \small{\texttt{s6lilemm@uni-bonn.de}}
    \and
    Martin Ueding \\ \small{\texttt{mu@martin-ueding.de}}
    \and
    Paul Manz \\ \small{\texttt{p.m@uni-bonn.de}}
}

\begin{document}
\maketitle
\section{Sattelpunktmethode}
\subsection{}
Gegeben ist das Integral
\begin{align}
    I &= \lim_{N\to\infty}\int_a^b\!\dif x\; \eexp{Nf(x)} \label{eq:integral}
    \intertext{$f$ hat ein globales Maximum bei $x=x_0$. Um diesen Punkt
    Taylorn wir die Funktion:}
    f(x) &= f\del{x_0} + f'\del{x_0}\del{x-x_0} +
    \frac{f''\del{x_0}}2\del{x-x_0}^2 + \dots
    \label{eq:taylor}
    \intertext{%
        Setzen wir \eqref{eq:taylor} in \eqref{eq:integral} ein und
        brechen nach dem dritten Glied ab, erhalten wir
    }
    I &= \lim_{N\to\infty}\int_a^b\!\dif x\;
    \eexp{Nf\del{x_0} + Nf'\del{x_0}x + \frac{N}2f''
    \del{x_0}\del{x-x_0}^2}\notag
    \intertext{%
        Die erste Ableitung verschwindet bei $x_0$, da dort das Maximum der
        Funktion ist.
    }
    &= \lim_{N\to\infty}\int_a^b\!\dif x\;
    \eexp{Nf\del{x_0}}\eexp{\frac{N}2f''\del{x_0}\del{x-x_0}^2}\notag \\
    &= \lim_{N\to\infty}\eexp{Nf\del{x_0}}\int_a^b\!\dif x\;
    \eexp{\frac{N}2f''\del{x_0}\del{x-x_0}^2}\notag \\
    &= \lim_{N\to\infty}\eexp{Nf\del{x_0}}\int_a^b\!\dif x\;
    \eexp{\frac N2f''\del{x_0}x^2}\eexp{Nf''\del{x_0}x_0x}
    \eexp{\frac N2f''\del{x_0}x_0^2} \notag \\
    &= \lim_{N\to\infty}\eexp{Nf\del{x_0}}\eexp{\frac N2f''\del{x_0}x_0^2}
    \int_a^b\!\dif x\;
    \eexp{\frac N2f''\del{x_0}x^2}\eexp{Nf''\del{x_0}x_0x} \notag
    \intertext{mit $a = -\frac N2 f''\del{x_0}$:}
    &= \lim_{N\to\infty} \eexp{Nf\del{x_0}} \eexp{-ax_0^2} \int_a^b\!\dif x\;
    \eexp{2ax_0x}\eexp{-ax^2} \notag
    \intertext{mit $\ii\omega = -2ax_0$}
    &= \lim_{N\to\infty} \eexp{Nf\del{x_0}} \eexp{\frac{\omega^2}{4a}} \int_a^b
    \!\dif x\; \eexp{-\ii\omega x}\eexp{-ax^2} \label{eq:integral_getaylort}
    \intertext{%
        Nun betrachten wir die Formel für Gaußsche Integrale
    }
    \sqrt{\frac{\piup}a}\eexp{-\frac{\omega^2}{4a}}&=
    \int\!\dif t\;\eexp{-\ii\omega t}\eexp{-at^2}\label{eq:Gauss} 
    \intertext{%
        Lösen wir nun \eqref{eq:integral_getaylort} mit \eqref{eq:Gauss}
        auf, erhalten wir:
    }
    &= \lim_{N\to\infty} \eexp{Nf\del{x_0}} \eexp{\frac{\omega^2}{4a}}
    \sqrt{\frac\piup a}\eexp{-\frac{\omega^2}{4a}}\notag \\
    &= \lim_{N\to\infty} \eexp{Nf\del{x_0}} \sqrt{\frac\piup a}\notag \\
    &= \lim_{N\to\infty} \eexp{Nf\del{x_0}}
    \sqrt{-\frac{2\piup}{Nf''\del{x_0}}}\notag
    \intertext{%
        Da die zweite Ableitung an der Maximalstelle negativ ist, folgt
    }
    &= \lim_{N\to\infty} \eexp{Nf\del{x_0}}
    \sqrt{\frac{2\piup}{N\abs{f''\del{x_0}}}}\label{eq:sattelpunktmethode}
\end{align}
\subsection{}

\begin{align*}
    \lim_{N\to\infty} N! &= \lim_{N\to\infty} \Gamma\del{N+1} \\
                         &= \lim_{N\to\infty} \int_0^\infty \!\dif x \;
    x^N\eexp{-x} \\
    \intertext{%
        Wir substituieren nun $x = Nz$ und $N \dif z = \dif x$:
    }
    &=\lim_{N\to\infty} \int_0^\infty \!\dif z \; NN^Nz^N \eexp{-Nz} \\
    &=\lim_{N\to\infty} NN^N \int_0^\infty \!\dif z \; 
    \eexp{N\log\del{z}} \eexp{-Nz} \\
    &=\lim_{N\to\infty} N^{N+1} \int_0^\infty \!\dif z \;
    \eexp{N\del{\log\del{z}-z}}
    \intertext{%
        Wir setzen nun $f(z) = \log\del z - z$. Diese Funktion hat ihr
        Maximum bei $z = 1$. Die zweite Ableitung an dieser Stelle ist
        $f''(1) = -\frac{1}{1^2} = -1$. Nach \eqref{eq:sattelpunktmethode}
        gilt
    }
    &= \lim_{N\to\infty} N^{N+1}\eexp{-N}\sqrt{\frac{2\piup}{N}}\\
    &=\lim_{N\to\infty}\sqrt{2\piup N}N^N\eexp{-N}
\end{align*}
Dies ist die gesuchte Stirling-Formel.

\section{Ensemble quantenmechanischer harmonischer Oszillatoren}

\subsection{Anzahl der Zustände}

Bei der Energie sind immer $N/2$ (in Einheiten von $\hbar\omega$ durch die $+1$ fest. Der Operator $a_j^\dagger a_j$ liefert den Eigenwert $n$. Somit müssen wir noch eine Energie von 3 auf drei Oszillatoren aufteilen. Es gibt folgende Möglichkeiten:
\[
    (0, 0, 3),\:
    (0, 1, 2),\:
    (0, 2, 1),\:
    (0, 3, 0),\:
    (1, 0, 2),\:
    (1, 1, 1),\:
    (1, 2, 0),\:
    (2, 0, 1),\:
    (2, 1, 0),\:
    (3, 0, 0)
\]

Es gibt also 10 Zustände.

\subsection{Wahrscheinlichkeit der einzelnen Zustände}

Der Zustand mit der Energie 0 kommt für jeden Oszillator viermal vor, also mit
einer Wahrscheinlichkeit \num{.4}. Die Energie 1 kommt dreimal, die Energie 2
zweimal und die Energie 3 einmal vor.

\subsection{$\Omega(E)$}

Für die Bearbeitung habe ich in \cite{Schwabl/Statistische_Mechanik} gelesen
und bin dabei in \cite[Abschnitt~2.2.3.1]{Schwabl/Statistische_Mechanik} auf
diese Aufgabe gestoßen.

Als eine Definition von $\Omega(E)$ habe ich Folgendes gefunden: \parencite[Formel~2.2.4]{Schwabl/Statistische_Mechanik}
\[
    \Omega(E) = \int \dif \Gamma \, \delta(E - H(q, p))
\]

Wobei $\Gamma$ den Phasenraum bezeichnet.

Der Phasenraum ist hier diskret und durch $\mathbb N^N$ gegeben. Daher brauchen wir hier anstelle vom Integral eine mehrdimensionale Summe über alle Energieniveaus:
\[
    \Omega(E) =
    \sum_{n_1=0}^\infty
    \ldots
    \sum_{n_N=0}^\infty
    \delta\del{
        E - \hbar\omega \sum_j \del{a_j^\dagger a_j + \frac12}
    }
\]

Was mich an dieser Stelle jedoch etwas verwundert ist, dass wir über alle $n_i$ jeweils summieren, obwohl diese explizit nicht in \cite[Formel~2.2.25]{Schwabl/Statistische_Mechanik} aufgeführt sind.

\subsection{Umschreiben}

Nun benutzen wir die Integraldarstellung der $\delta$-Distribution:\parencite[Formel~2.2.4]{Schwabl/Statistische_Mechanik}
\[
    \delta(y) = \int \frac{\dif k}{2 \pi} \eup^{\iup k y}
\]

Somit erhalten wir:
\[
    \Omega(E) =
    \sum_{n_1=0}^\infty
    \ldots
    \sum_{n_N=0}^\infty
    \int \frac{\dif k}{2 \pi}
    \exp\sbr{
        \iup k \del{
            E - \hbar\omega \sum_j \del{a_j^\dagger a_j + \frac12}
        }
    }
\]

Die Summe in der Exponentialfunktion können wir noch als Produkt von von Exponentialfunktionen schreiben
\[
    \Omega(E) =
    \sum_{n_1=0}^\infty
    \ldots
    \sum_{n_N=0}^\infty
    \int \frac{\dif k}{2 \pi}
    \eup^{\iup k E}
    \prod_{j=1}^N
    \eup^{-\iup k \hbar\omega /2}
    \exp( - \iup k \hbar\omega n_j )
\]

Für jede Summe $\sum$ ergibt sich in der letzten Exponentialfunktion eine Geometrische Reihe mit $q = \exp( - \iup k \hbar\omega)$. Da von 0 bis $\infty$ addiert wird, kann direkt der Grenzwert $(1 - q^\infty)/(1-q)$ eingesetzt werden:
\[
    \Omega(E) =
    \int \frac{\dif k}{2 \pi}
    \eup^{\iup k E}
    \prod_{j=1}^N
    \eup^{-\iup k \hbar\omega /2}
    \frac1{1 - \exp( - \iup k \hbar\omega)}
\]

Da das Produkt $\prod$ nur noch gleiche Faktoren multipliziert, kann es durch
eine Potenz ersetzt werden. Somit erhalten wir die gewünschte Form:
\[
    \Omega(E) =
    \int \frac{\dif k}{2 \pi}
    \eup^{\iup k E}
    \del{ \frac{\eup^{-\iup k \hbar\omega /2}}{1 - \exp( - \iup k \hbar\omega)} }^N
\]

Dies sollen wir jetzt noch ein wenig umschreiben. Dazu betrachten wir nur die Klammer:
\begin{align*}
    \del{ \frac{\eup^{-\iup k \hbar\omega /2}}{1 - \exp( - \iup k \hbar\omega)} }^N
    &=: \del{\frac{\eup^{-\beth/2}}{1 - \eup^{-\beth}}}^N \\
    &= \del{\frac{1 - \eup^{-\beth}}{\eup^{-\beth/2}}}^{-N} \\
    &= \del{\eup^{\beth/2} - \eup^{-\beth/2}}^{-N} \\
    &= \del{2\iup \sin(-\iup \beth/2)}^{-N} \\
    &= \exp\del{-N \ln\del{2\iup \sin(-\iup \beth/2)}} \\
    &= \exp\del{-N \ln\del{2\iup \sin(k \hbar\omega/2)}}
\end{align*}

Alles zusammen ergibt den gesuchten Ausdruck:
\[
    \Omega(E) =
    \int \frac{\dif k}{2 \pi}
    \exp\del{N \del{ \iup k \frac EN - \ln\del{2\iup \sin(k \hbar\omega/2)} }}
\]

\subsection{Berechnen des Integrals}

Dieses Integral kann so umgeschrieben werden, dass man es mit der Sattelpunktsmethode ausrechnen kann. Dabei ist:
\[
    f(k) = \iup k \frac EN - \ln\del{2\iup \sin(k \hbar\omega/2)}
\]

Wir bestimmen die erste und zweite Ableitung, da wir diese im weiteren Verlauf brauchen werden.
\[
    f'(k) = \iup \frac EN - \cot(k \hbar\omega/2) \frac{\hbar\omega}{4}
\]

Die zweite Ableitung ist:
\[
    f''(k) = - \frac{\hbar^2 \omega^2}8 \csc^2\del{\frac{k \hbar\omega}2}
\]

Laut \cite[Abschnitt~2.2.3.1]{Schwabl/Statistische_Mechanik} ist es jedoch 4 anstelle von $-8$. Wir benutzen das Ergebnis aus dem Buch.

Die Nullstelle der ersten Ableitung ist bei:
\[
    k_0 = \frac{1}{\iup\hbar\omega} \ln\del{\frac{2e + \frac{\hbar\omega}2}{2e - \frac{\hbar\omega}2}}
\]

Dabei haben wir $e := E/N$ eingesetzt und die Darstellung des $\arctan$ mit dem
komplexen $\ln$ ausgenutzt.

In \cite[Formel~2.2.28]{Schwabl/Statistische_Mechanik} ist es nicht $2e$ sondern nur $e$. Wir rechnen mit dem Ergebnis aus dem Buch weiter.

Nun müssen wir die Nullstelle in die zweite Ableitung einsetzen.
\begin{align*}
    f''(k_0)
    &= \frac{\hbar^2 \omega^2}4 \csc^2\del{\frac{k_0 \hbar\omega}2} \\
    &= \frac{\hbar^2 \omega^2}4 \csc^2\sbr{\frac{\hbar\omega}2\frac{1}{\iup\hbar\omega} \ln\del{\frac{e + \frac{\hbar\omega}2}{e - \frac{\hbar\omega}2}}} \\
    &= \frac{\hbar^2 \omega^2}4 \csc^2\sbr{\frac{1}{2\iup} \ln\del{\frac{e + \frac{\hbar\omega}2}{e - \frac{\hbar\omega}2}}} \\
    \intertext{%
        Jetzt können wir dies wieder als $\arctan$ schreiben:
    }
    &= \frac{\hbar^2 \omega^2}4 \csc^2\sbr{ \arctan\del{\frac{\hbar\omega}{\iup e}}} \\
    \intertext{%
        Laut Mathematica ist $\csc^2(\arctan(x))$ gerade $1 + 1/x^2$.
    }
    &= \frac{\hbar^2 \omega^2}4 \del{1 - \frac{e^2}{\hbar^2\omega^2}} \\
    &= \frac{\hbar^2 \omega^2 - e^2}4
\end{align*}

Jetzt müssen wir das in die Wurzel einsetzen und erhalten:
\[
    \sqrt{\frac{8 \pi}{N (\hbar^2 \omega^2 - e^2}}
\]

Es fehlt noch der Rest. Alles zusammen ist dann:
\[
    \Omega(E) = \lim_{N \to \infty} \eup^{N f(k_0)} \sqrt{\frac{8 \pi}{N (\hbar^2 \omega^2 - e^2}}
\]

Noch $f(k_0)$ einsetzen:
\[
    \Omega(E) = \lim_{N \to \infty}
    \exp\cbr{
        N 
        \del{
            \iup k \frac EN - \ln\sbr{
                2\iup \sin\del{
                    \cbr{
                        \frac{1}{\iup\hbar\omega}
                        \ln\sbr{
                            \frac{e + \frac{\hbar\omega}2}{e - \frac{\hbar\omega}2}
                        }
                    }
                    \hbar\omega/2
                }
            }
        }
    }
    \sqrt{\frac{8 \pi}{N (\hbar^2 \omega^2 - e^2)}}
\]

Da sehe ich allerdings nicht, wie das auf die gewünschte Form zu bringen ist.

\printbibliography

\end{document}
