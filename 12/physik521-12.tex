% Für Seitenformatierung

\documentclass[DIV=15]{scrartcl}

% Zeilenumbrüche

\parindent 0pt
\parskip 6pt

% Für deutsche Buchstaben und Synthax

\usepackage[ngerman]{babel}

% Für Auflistung mit speziellen Aufzählungszeichen

\usepackage{paralist}

% zB für \del, \dif und andere Mathebefehle

\usepackage{amsmath}
\usepackage{commath}
\usepackage{amssymb}

% für nicht kursive griechische Buchstaben

\usepackage{txfonts}

% Für \SIunit[]{} und \num in deutschem Stil

\usepackage[output-decimal-marker={,}]{siunitx}
\usepackage[utf8]{inputenc}

% Für \sfrac{}{}, also inline-frac

\usepackage{xfrac}

% Für Einbinden von pdf-Grafiken

\usepackage{graphicx}

% Umfließen von Bildern

\usepackage{floatflt}

% Für Links nach außen und innerhalb des Dokumentes

\usepackage{hyperref}

% Für weitere Farben

\usepackage{color}

% Für Streichen von z.B. $\rightarrow$

\usepackage{centernot}

% Für Befehl \cancel{}

\usepackage{cancel}

% Für Layout von Links

\hypersetup{
	citecolor=black,
	colorlinks=true,
	linkcolor=black,
	urlcolor=blue,
}

% Verschiedene Mathematik-Hilfen

\newcommand \e[1]{\cdot10^{#1}}
\newcommand\p{\partial}

\newcommand\half{\frac 12}
\newcommand\shalf{\sfrac12}

\newcommand\skp[2]{\left\langle#1,#2\right\rangle}
\newcommand\mw[1]{\left\langle#1\right\rangle}
\renewcommand \exp[1]{\mathrm e^{#1}}

% Nabla und Kombinationen von Nabla

\renewcommand\div[1]{\skp{\nabla}{#1}}
\newcommand\rot{\nabla\times}
\newcommand\grad[1]{\nabla#1}
\newcommand\laplace{\triangle}
\newcommand\dalambert{\mathop{{}\Box}\nolimits}

%Für komplexe Zahlen

\renewcommand \i{\mathrm i}
\renewcommand{\Im}{\mathop{{}\mathrm{Im}}\nolimits}
\renewcommand{\Re}{\mathop{{}\mathrm{Re}}\nolimits}

%Für Bra-Ket-Notation

\newcommand\bra[1]{\left\langle#1\right|}
\newcommand\ket[1]{\left|#1\right\rangle}
\newcommand\braket[2]{\left\langle#1\left.\vphantom{#1 #2}\right|#2\right\rangle}
\newcommand\braopket[3]{\left\langle#1\left.\vphantom{#1 #2 #3}\right|#2\left.\vphantom{#1 #2 #3}\right|#3\right\rangle}

\newcommand{\eqnsep}{,\quad}


%\subject{}
\title{physik521 – Übung 12}
%\subtitle{}
\author{
	Martin Ueding \\ \small{\href{mailto:mu@martin-ueding.de}{mu@martin-ueding.de}}
        \and Paul Manz
        \and Lino Lemmer \\ \small{\href{mailto:l2@uni-bonn.de}{l2@uni-bonn.de}}
}

\pagestyle{plain}

\newcommand\kB{k_\text B}
\newcommand\muB{\mu_\text B}
\newcommand\ZG{Z_\text G}

\begin{document}

\maketitle

\section{Thermisches Photonengas}

Gegeben ist:
\[
    \rho(E) = \frac{V E^2}{\piup (\hbar c)^3}
    \eqnsep
    E(\vec p) = |\vec p| c.
\]

\subsection{Großkanonisches Potential}

Die Photonen interagieren nicht miteinander. Also sollte es keinen
Energieunterschied ausmachen, wenn sich die Anzahl ändert. Daher ist $\mu = 0$.

Die Zustände, die wir haben, sind die verschiedenen Wellenvektoren $\vec k \in
\R^3$. Diese hängen mit dem Impulsen $\vec p$ über $\vec p = \hbar \vec k$
zusammen. Als weitere Freiheitsgerade haben wir die beiden Helizitäten $\pm 1$.
Somit ist die großkanonische Zustandssumme:
\begin{align*}
    \ZG
    &= \sum_{n=0}^\infty \exp\del{- \beta \sum_{\vec k \in \R^3} \sum_{s=\pm1} E(\vec k)} \\
    &= \sum_{n=0}^\infty \prod_{\vec k \in \R^3} \prod_{s=\pm1} \exp(- \beta E(\vec k)) \\
    \intertext{%
        Wir wenden die Formel für den Grenzwert der geometrischen Reihe an.
        Außerdem führen wir das Produkt über die beiden Helizitäten aus.
    }
    &= \del{\prod_{\vec k \in \R^3} \frac{1}{1 - \exp(- \beta E(\vec k))}}^2 \\
    &=: \del{\prod_{\vec k \in \R^3} \tilde Z_\text G}^2
\end{align*}
    
Daraus bestimmen wir jetzt das großkanonische Potential:
\begin{align*}
    \Omega
    &= - \kB T \ln(\ZG) \\
    &= - 2 \kB T \sum_{\vec k \in \R^3} \ln(\tilde Z_\text G (\vec k)) \\
    \intertext{%
        An dieser Stelle müssen wir die Summe in ein Integral umschreiben, um
        weiter rechnen zu können. Die Wellenzahlzustände (Impulszustände) sind
        aufgrund der am Volumenrand verschwindenden Wellenfunktion quantisiert.
        Wir nehmen ohne Beschränkung der Allgemeinheit an, dass das Volumen ein
        Würfel mit Kantenlänge $l$ ist. Es gibt eine kleinste Wellenzahl $k_0$,
        die durch $2\piup/l$ gegeben ist. Alle höheren Wellenzahlen sind
        Vielfache davon. Somit schreiben wir die Summe in einer Dimension um
        als $\sum_{k_x \in \R} = \sum_{n \in \Z} k_0 n$. Aus der Summe über $n$
        können wir nun ein Integral $\int \dif n$ machen, da $l$ so groß ist,
        dass $n$ quasidicht in $\R$ liegt. Mit $\dif n = \dif k l / (2\piup)$
        erhalten wir einen Vorfaktor $V/(2\piup)^3$ vor dem Integral, wenn wir die Summe
        in allen drei Dimensionen in ein Integral umwandeln.
    }
    &= - 2 \kB T \frac{V}{(2\piup)^3} \int_{\R^3} \dif{^3 k} \, \ln(\tilde Z_\text G (\vec k)) \\
    \intertext{%
        Um die normale Darstellung mit $\vec p$ zu erhalten, substituieren wir
        erneut.
    }
    &= - 2 \kB T \frac{V}{(2\piup\hbar)^3} \int_{\R^3} \dif{^3 p} \, \ln\del{\frac{1}{1 -
    \exp(- \beta |\vec p|c)}} \\
    &= 2 \kB T \frac{V}{(2\piup\hbar)^3} \int_{\R^3} \dif{^3 p} \, \ln\del{1 -
    \exp(- \beta |\vec p|c)} \\
    \intertext{%
        Da wir nur am Betrag von $\vec p$ interessiert sind, wählen wir
        Kugelkoordinaten.
    }
    &= 2 \frac{4 \piup \kB T V}{(2\piup\hbar)^3} \int_{0}^\infty \dif p \, p^2
    \ln\del{1 - \exp(- \beta pc)} \\
    \intertext{%
        Nun substituieren wir, damit wir den Tipp auf dem Aufgabenblatt
        anwenden können, $x := \beta pc$.
    }
    &= 2 \frac{4 \piup \kB T V}{(2\piup\hbar)^3 (\beta c)^3} \int_{0}^\infty
    \dif x \, x^2 \ln\del{1 - \exp(-x)} \\
    \intertext{%
        Das verbleibende Integral ist laut Tipp $-\piup^4/45$.
    }
    &= - \frac{\piup^2 \kB^4 V}{45 (\hbar c)^3} T^4
\end{align*}

Dies stimmt mit dem Kontrollergebnis überein.

\subsection{Mittlere Photonenzahl}

Die Anzahl der Teilchen ist die Summe der Besetzungszahlen der Zustände. So ist
im Skript als Formel (5.136) angegeben:
\[
    N = \sum_\alpha b(E_\alpha).
\]

Dort wird aber über die Zustände $\alpha$ summiert. Wir möchten lieber über die
Energien summieren und gebrauchen daher die schon gegebene Zustandsdichte:
\begin{align*}
    N
    &= \int \dif E \, \rho(E) b(E) \\
    &= \dif E \frac{V E^2}{\piup^2 \hbar^3 c^3} \frac{1}{\exp(-\beta E) - 1} \\
    \intertext{%
        Hier können wir das Integral auf dem Aufgabenblatt benutzen, wenn wir
        $x := \beta E$ setzen.
    }
    &= V \del{\frac{\kB T}{\hbar c}}^3 \int \dif x \, x^2 \frac{1}{\exp(-x) -1
} \\
    &\approx \num{0.244} \del{\frac{\kB T}{\hbar c}}^3 V
\end{align*}

Auch dies stimmt mit dem Kontrollergebnis überein.

\subsection{Entropie und innere Energie}

Entropie:
\begin{align*}
    S
    &= - \dpd\Omega T \\
    &= \frac{4 \piup^2 \kB^4 V}{45 \hbar^3 c^3} T^3
\end{align*}

Innere Energie:
\[
    U = \omega + TS = (4 + 1) \frac{\piup^2 \kB^4 V}{45 \hbar^3 c^3} T^4
\]

Da $\Omega = U - TS - \mu N$ sowie $F = U - TS$ ist, sind diese beiden
Potentiale identisch, sobald $\mu = 0$ ist.

\subsection{Druck}

Es gilt:
\[
    \dif \Omega = - S \dif T - p \dif V - N \dif \mu
\]

Daher:
\[
    p = - \tdpd \Omega V {T, \mu}
    = \frac{\piup^2 \kB^4}{45 \hbar^3 c^3} T^4
\]

Der Druck hängt nicht vom Volumen ab. Der Strahlungsdruck steigt also nicht
direkt durch alleinige Verringerung des Volumens. Dies liegt daran, dass die
Photonen nicht wechselwirken.

\subsection{Adiabatische Zustandsgleichungen}

Moment! Adiabatisch bedeutet $\deltaup Q = 0$. $\dif S = 0$ bedeutet
reversibel. Wir gehen hier davon aus, dass $\dif S = 0$ das entscheidende ist.

Nach „Einführung in die Extragalaktische Astronomie“ erwarten wir, dass die
Energiedichte mit einer Längenskala $l$ in der Form $l^{-4}$ skaliert.

\begin{align*}
    S &= \frac{4 \pi^2 \kB^4 V}{45 \hbar^3 c^3} T^3 \\
    \dif S &= \frac{4 \pi^2 \kB^4}{45 \hbar^3 c^3} (\dif V \, T^3 + 3 V T^2 \dif
    T) \overset != 0 \\
    \iff 0 &= \dif V \, T + 3 V \dif T
\end{align*}

$N$ hängt ebenfalls von $V T^3$ ab, hat also ein totales Differential $\dif N$,
das proportional zu $\dif S$ ist. Somit ist $\dif N = 0$ und die Teilchenzahl
erhalten.

Wenn $N$ eine Konstante ist, dann kann man daraus $T(V)$ erhalten, in dem man
$N(T, V)$ umstellt:
\[
    T(V) = \frac{\hbar c}\kB \sqrt[3]{\frac{N}{\num{0.244} V}}.
\]

Die Temperatur nimmt also mit dem Volumen ab. Die Energie hängt von der
Temperatur und dem Volumen mit $U \propto V T^4$ ab. Somit die Energiedichte
$\rho_U \propto V^{-3/4} = l^{-4}$. Dies stimmt mit der Friedmann-Gleichung
überein.

Für
den Druck setzen wir $T(V)$ ein und erhalten $p(V)$:
\[
    p = \frac{\piup^2 \kB^4}{45 \hbar^3 c^3} T^4
\]

\section{Klassisches Gas in einem harmonischen Potential}

\subsection{Kanonische Zustandssumme}

\begin{align*}
    Z_\text C
    &= \sum_{\vec p_1, \vec x_1} \ldots \sum_{\vec p_N, \vec x_N} \exp(-\beta
    H) \\
    &= \del{\sum_{\vec p, \vec x} \exp(-\beta H)}^N \\
    &= \del{\frac{V}{(2\piup\hbar)^2} \int_{\R^3} \dif{^3 p} \int_{\R^3}
    \dif{^3 x} \, \exp\del{-\beta
\del{\frac{\vec p^2}{2m} + \kappa \vec x^2}}}^N \\
\intertext{%
Wir benutzen wieder Kugelkoordinaten.
}
    &= \del{\frac{16 \piup^2 V}{(2\piup\hbar)^2} \int_0^\infty \dif p
\int_0^\infty \dif x \, p^2 x^2 \exp\del{-\beta
\del{\frac{p^2}{2m} + \kappa x^2}}}^N \\
\intertext{%
    Ein derartiges Integral hatten wir schon auf einem vorherigen
    Aufgabenzettel. Wir haben dieses jetzt mit Mathematica gelöst.
}
&= \del{\frac{V}{\sqrt 8 \hbar^3 \beta^3} \del{\frac m \kappa}^{3/2}}^N
\end{align*}

Daraus bestimmen wie die freie Energie zu:
\[
    F = - \kB T N \ln\del{\frac{V}{\sqrt 8 \hbar^3 \beta^3} \del{\frac m \kappa}^{3/2}}
\]

\subsection{Mittlerer Quadratischer Radius}

\[
    \bracket{x^2} = \frac{1}{Z_\text C} \sum_\alpha \exp(- \beta H) x^2
\]

Somit ändert sich das Integral, das wir in der vorherigen Aufgabe ausgerechnet
haben, nur durch den weiteren Faktor $x^2$. Und natürlich wird $Z_\text C$
herausgeteilt. Mathematica liefert uns den Ausdruck:
\[
    R = \frac 32 \frac{\kB T}{\kappa}.
\]

Von den Einheiten stimmt dies auch.

Das Volumen nehmen wir als Kugelförmig an, wodurch wir erhalten:
\[
    V = \frac 43 \piup \del{\frac 32 \frac{\kB T}\kappa}^{3/2}.
\]

\subsection{Zustandssumme und freie Energie als Funktion von $N$, $V$ und $T$}

Da wir die Integrale in den vorherigen Aufgabenteilen schon ausgerechnet haben,
brauchen wir in dieser Aufgabe nichts mehr tun.

\subsection{Druck}

Allgemein gilt:
\[
    F = - pV + \mu N
    \iff
    p = - \tdpd F V {N, T}
\]

Hier ist:
\[
    F = - \kB T N \ln\del{\frac{V}{\sqrt 8 (\hbar \beta)^3} \del{\frac{m}
        \kappa}^{3/2}}
    \]

Nach Vereinfachungen erhalten wir für den Druck:
\[
    p(N, V, T) = \frac{\kB T N}{V}
\]

Dies ist allerdings nur eine Funktion, wir wissen nicht, wie wir daraus eine
Zustands\emph{gleichung} erhalten.

$p(V)$ ist eine einfache Hyperbel. Wie kann sich denn $V$ ändern, wenn $N$ und
$T$ konstant gehalten werden? Ändern sich dann $m$ oder $\kappa$?


\IfFileExists{\bibliographyfile}{
    \printbibliography
}{}

\end{document}

% vim: spell spelllang=de tw=79
